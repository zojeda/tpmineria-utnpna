\RequirePackage{fixltx2e}
\documentclass[runningheads,a4paper]{llncs}

\usepackage[american]{babel}
\usepackage{booktabs}
\usepackage{longtable}
\usepackage{array}
\usepackage{multirow}
\usepackage[table]{xcolor}
\usepackage{wrapfig}
\usepackage{float}
\usepackage{colortbl}
\usepackage{pdflscape}
\usepackage{tabu}
\usepackage{threeparttable}
\usepackage{threeparttablex}
\usepackage[normalem]{ulem}
\usepackage{makecell}
\usepackage{graphicx}
\usepackage{biblatex}
%extended enumerate, such as \begin{compactenum}
\usepackage{paralist}

%put figures inside a text
%\usepackage{picins}
%use
%\piccaptioninside
%\piccaption{...}
%\parpic[r]{\includegraphics ...}
%Text...

%Sorts the citations in the brackets
%\usepackage{cite}

%for easy quotations: \enquote{text}
\usepackage{csquotes}

\usepackage[T1]{fontenc}

%better font, similar to the default springer font
\usepackage{lmodern}
%if more space is needed, exchange lmodern by mathptmx
%\usepackage{mathptmx}

%enable margin kerning
\usepackage{microtype}

%for demonstration purposes only
\usepackage[math]{blindtext}

\usepackage{ifxetex,ifluatex}
\ifxetex
  \usepackage{fontspec,xltxtra,xunicode}
  \defaultfontfeatures{Mapping=tex-text,Scale=MatchLowercase}
  \newcommand{\euro}{€}
\else
  \ifluatex
    \usepackage{fontspec}
    \defaultfontfeatures{Mapping=tex-text,Scale=MatchLowercase}
    \newcommand{\euro}{€}
  \else
    \usepackage[utf8]{inputenc}
    \usepackage{eurosym}
  \fi
\fi


\makeatletter
\renewcommand\subsubsection{\@startsection{subsubsection}{3}{\z@}%
                       {-18\p@ \@plus -4\p@ \@minus -4\p@}%
                       {4\p@ \@plus 2\p@ \@minus 2\p@}%
                       {\normalfont\normalsize\bfseries\boldmath
                        \rightskip=\z@ \@plus 8em\pretolerance=10000 }}
\renewcommand\paragraph{\@startsection{paragraph}{4}{\z@}%
                       {-12\p@ \@plus -4\p@ \@minus -4\p@}%
                       {2\p@ \@plus 1\p@ \@minus 1\p@}%
                       {\normalfont\normalsize\itshape
                        \rightskip=\z@ \@plus 8em\pretolerance=10000 }}
\makeatother


%\usepackage[capitalise,nameinlink]{cleveref}
%Nice formats for \cref
%\crefname{section}{Sect.}{Sect.}
%\Crefname{section}{Section}{Sections}
%\crefname{figure}{Fig.}{Fig.}
%\Crefname{figure}{Figure}{Figures}

\usepackage{xspace}
%\newcommand{\eg}{e.\,g.\xspace}
%\newcommand{\ie}{i.\,e.\xspace}
\newcommand{\eg}{e.\,g.,\ }
\newcommand{\ie}{i.\,e.,\ }

% correct bad hyphenation here
\hyphenation{op-tical net-works semi-conduc-tor}

%%%%%%%%%%%%%%%%%%%%%%%%%%%%%%%%%%%%%%%%%%%%%%%%
\usepackage{fancyvrb}
%\DefineShortVerb[commandchars=\\\{\}]{\|}
\DefineVerbatimEnvironment{Highlighting}{Verbatim}{commandchars=\\\{\},fontsize=\footnotesize}
% Add ',fontsize=\small' for more characters per line
\newenvironment{Shaded}{\footnotesize}{}
\newcommand{\KeywordTok}[1]{\textcolor[rgb]{0.00,0.44,0.13}{\textbf{{#1}}}}
\newcommand{\DataTypeTok}[1]{\textcolor[rgb]{0.56,0.13,0.00}{{#1}}}
\newcommand{\DecValTok}[1]{\textcolor[rgb]{0.25,0.63,0.44}{{#1}}}
\newcommand{\BaseNTok}[1]{\textcolor[rgb]{0.25,0.63,0.44}{{#1}}}
\newcommand{\FloatTok}[1]{\textcolor[rgb]{0.25,0.63,0.44}{{#1}}}
\newcommand{\CharTok}[1]{\textcolor[rgb]{0.25,0.44,0.63}{{#1}}}
\newcommand{\StringTok}[1]{\textcolor[rgb]{0.25,0.44,0.63}{{#1}}}
\newcommand{\CommentTok}[1]{\textcolor[rgb]{0.38,0.63,0.69}{\textit{{#1}}}}
\newcommand{\OtherTok}[1]{\textcolor[rgb]{0.00,0.44,0.13}{{#1}}}
\newcommand{\AlertTok}[1]{\textcolor[rgb]{1.00,0.00,0.00}{\textbf{{#1}}}}
\newcommand{\FunctionTok}[1]{\textcolor[rgb]{0.02,0.16,0.49}{{#1}}}
\newcommand{\RegionMarkerTok}[1]{{#1}}
\newcommand{\ErrorTok}[1]{\textcolor[rgb]{1.00,0.00,0.00}{\textbf{{#1}}}}
\newcommand{\NormalTok}[1]{{#1}}
\newcommand{\ConstantTok}[1]{\textcolor[rgb]{0.53,0.00,0.00}{{#1}}}
\newcommand{\SpecialCharTok}[1]{\textcolor[rgb]{0.25,0.44,0.63}{{#1}}}
\newcommand{\VerbatimStringTok}[1]{\textcolor[rgb]{0.25,0.44,0.63}{{#1}}}
\newcommand{\SpecialStringTok}[1]{\textcolor[rgb]{0.73,0.40,0.53}{{#1}}}
\newcommand{\ImportTok}[1]{{#1}}
\newcommand{\DocumentationTok}[1]{\textcolor[rgb]{0.73,0.13,0.13}{\textit{{#1}}}}
\newcommand{\AnnotationTok}[1]{\textcolor[rgb]{0.38,0.63,0.69}{\textbf{\textit{{#1}}}}}
\newcommand{\CommentVarTok}[1]{\textcolor[rgb]{0.38,0.63,0.69}{\textbf{\textit{{#1}}}}}
\newcommand{\VariableTok}[1]{\textcolor[rgb]{0.10,0.09,0.49}{{#1}}}
\newcommand{\ControlFlowTok}[1]{\textcolor[rgb]{0.00,0.44,0.13}{\textbf{{#1}}}}
\newcommand{\OperatorTok}[1]{\textcolor[rgb]{0.40,0.40,0.40}{{#1}}}
\newcommand{\BuiltInTok}[1]{{#1}}
\newcommand{\ExtensionTok}[1]{{#1}}
\newcommand{\PreprocessorTok}[1]{\textcolor[rgb]{0.74,0.48,0.00}{{#1}}}
\newcommand{\AttributeTok}[1]{\textcolor[rgb]{0.49,0.56,0.16}{{#1}}}
\newcommand{\InformationTok}[1]{\textcolor[rgb]{0.38,0.63,0.69}{\textbf{\textit{{#1}}}}}
\newcommand{\WarningTok}[1]{\textcolor[rgb]{0.38,0.63,0.69}{\textbf{\textit{{#1}}}}}
\ifxetex
  \usepackage[setpagesize=false, % page size defined by xetex
              unicode=false, % unicode breaks when used with xetex
              xetex,
              colorlinks=true,
              linkcolor=blue]{hyperref}
\else
%  \usepackage[unicode=true,
%              colorlinks=true,
%              linkcolor=blue]{hyperref}
    %unobstrusive usage of hyperref
    \ifnum\pdfoutput>0
        \usepackage[
        unicode=true,
        %pdfauthor={},
        %pdfsubject={},
        %pdftitle={},
        %pdfkeywords={},
        bookmarks=false,
        breaklinks=true,
        colorlinks=true,
        linkcolor=black,
        citecolor=black,
        urlcolor=black,
        %pdfstartpage=19,
        pdfpagelayout=SinglePage
        ]{hyperref}
        %enables correct jumping to figures when referencing
        \usepackage[all]{hypcap}
    \else
        \usepackage{hyperref}
    \fi
\fi
\hypersetup{breaklinks=true, pdfborder={0 0 0}}
\setlength{\parindent}{15pt} % set to 0pt if you want no indent in the first line of a paragraph
%\setlength{\parskip}{6pt plus 2pt minus 1pt}
\setlength{\parskip}{0pt}
\setlength{\emergencystretch}{3em}  % prevent overfull lines
\setcounter{secnumdepth}{3}
%\EndDefineVerbatimEnvironment{Highlighting}

\raggedbottom % allow ragged page bottoms

\def\tightlist{} % fix error when translating md lists to itemize

%%%%%%%%%%%%%%%%%%%%%%%%%%%%%%%%%%%%%%%%%%%%%%%%

\begin{document}


\title{Minería de Datos}


%If Title is too long, use \titlerunning
%\titlerunning{Short Title}
%Single insitute
\author{Bodean, Emiliano - Ojeda, Zacarías}

%If there are too many authors, use \authorrunning
%\authorrunning{First Author et al.}


\institute{
UTN Regional Paraná
}

%Multiple insitutes
%Currently disabled
%
\iffalse
\authorinfo{
  
}{
}{
  \{\}
}
\fi


\maketitle


	\begin{abstract}
		Se presenta un análisis sobre los tiempos entre inicio y sentencia de
causas judiciales, en la búsqueda de identificar patrones o asociaciones
que lleven a diferentes tiempos de demora en los mismos.
	\end{abstract}

	\keywords{Minería de Datos, sentencia, justicia, demora, reglas de asociación}

\section{Introducción}\label{introduccion}

En el trabajo se realiza un estudio sobre sentencias judiciales, en el
análisis se pretende evaluar los tiempos de demora de la sentencias.
Detectando patrones comunes, o asociaciones, que resulten significativas
con estos tiempo de demora.

Se considera la demora como el tiempo entre que inicia el proceso y la
sentencia correspondiente que pone fin al conflicto.

Se cuenta con una base de datos de resoluciones correspondientes al
Superior Tribunal de Justicia de la Provincia de Entre Ríos, que a sido
debidamente anonimizada, quitando referencias a las partes
intervinientes y carátulas de las causas.

El análisis se realiza analizando demoras según el tipo de proceso,
debido a que cada tipo de proceso, por su naturaleza, implica diferente
tratamiento por parte de los organismos (juzgados) y por ende supone
demoras esperables diferentes.

El procesamiento y análisis de datos se ha realizado con el
\href{https://www.r-project.org}{lenguaje R} v3.4.4 \autocite{R},
utilizando \href{}{dplyr} para preprocesamiento de datos,
\href{}{arules} para creación de reglas de asociación y
\href{}{arulesViz} para visualizarlas. El presente informe se ha
realizado en Rmd para producir resultados reproducibles XXX cita.

\section{Procesamiento de Datos}\label{procesamiento-de-datos}

\subsection{Lectura de Datasets}\label{lectura-de-datasets}

Cargamos los datasets originales. Contamos con un listado de 25279
sentencias las cuales tiene los siguientes datos de interés:

\begin{itemize}
\tightlist
\item
  Tipo de Proceso.
\item
  Fecha de Inicio y Fecha de Resolución.
\item
  Cantidad de Justiciable.
\item
  Circunscripción.
\item
  Materia.
\item
  Capital Reclamado.
\item
  Organismos.
\end{itemize}

Este último campo, organismos, esta con un código interno (iep) por lo
que se importa otra tabla para traducir este código.

\begin{Shaded}
\begin{Highlighting}[]
\NormalTok{sentencias_1c <-}\StringTok{ }\KeywordTok{read_csv}\NormalTok{(}\StringTok{"./sentencias_1c.csv"}\NormalTok{) }\OperatorTok\StringTok{ }
\StringTok{  }\KeywordTok{filter}\NormalTok{(}\KeywordTok{is.na}\NormalTok{(mat) }\OperatorTok{|}\StringTok{ }\KeywordTok{toupper}\NormalTok{(mat)}\OperatorTok{==}\StringTok{"C"}\NormalTok{) }\OperatorTok\StringTok{ }
\StringTok{  }\CommentTok{# solo incluir materia civil  }
\StringTok{  }\KeywordTok{select}\NormalTok{(}\OperatorTok{-}\NormalTok{mat) }

\NormalTok{organismos <-}\StringTok{ }\KeywordTok{read_csv}\NormalTok{(}\StringTok{"./organismos.csv"}\NormalTok{)}
\end{Highlighting}
\end{Shaded}

\subsection{Inspección y categorización de los
datos}\label{inspeccion-y-categorizacion-de-los-datos}

A continuación hacemos una inspección de los datos para poder
categorizarlos en rangos que nos sean útiles para la generación de
reglas de asociación.

En las siguientes tablas se ven fragmentos de las tablas de sentencias y
organismos sin procesar.

\rowcolors{2}{gray!6}{white}

\begin{table}

\caption{\label{tab:unnamed-chunk-3}Sentencias Primera Instancia Original}
\centering
\resizebox{\linewidth}{!}{
\begin{tabular}[t]{l|l|l|r|l|l|l|l|l|r|r|r|l}
\hiderowcolors
\hline
nro & tproc & as & ccon & finicio & fdesp & fvenc1 & fvenc2 & fres & tres & justiciables & reccap & iep\\
\hline
\showrowcolors
12858 & RESTRICCIONES A LA CAPACIDAD & S & 0 & 15/09/2015 & 28/11/2017 & 22/12/2017 & 19/02/2018 & 29/12/2017 & 7 & 1 & 0 & jdofam0002gch\\
\hline
11852/5 & INCIDENTE & S & 1 & 15/04/2015 & 19/09/2017 & 04/10/2017 & 19/10/2017 & 29/12/2017 & 6 & 3 & 0 & jdofam0002gch\\
\hline
12237 & ORDINARIO FILIACION E INDEMNIZACION DE DAÑOS & S & 1 & 03/12/2014 & 12/10/2017 & 13/12/2017 & 19/03/2018 & 29/12/2017 & 7 & 1 & 0 & jdofam0002gch\\
\hline
14440 & MEDIDA CAUTELAR (FAMILIA) & S & 0 & 21/04/2017 & 29/11/2017 & 18/12/2017 & 02/02/2018 & 29/12/2017 & 7 & 1 & 0 & jdofam0002gch\\
\hline
11507 & ORDINARIO DAÑOS Y PERJUICIOS & S & 1 & 13/03/2014 & 30/11/2017 & 02/02/2018 & 06/04/2018 & 29/12/2017 & 7 & 2 & 0 & jdofam0002gch\\
\hline
8133 & ORDINARIO FILIACION E INDEMNIZACION DE DAÑOS & S & 1 & 17/06/2010 & 16/03/2017 & 17/05/2017 & 28/07/2017 & 29/12/2017 & 7 & 2 & 0 & jdofam0002gch\\
\hline
\end{tabular}}
\end{table}

\rowcolors{2}{white}{white}

\rowcolors{2}{gray!6}{white}

\begin{table}

\caption{\label{tab:unnamed-chunk-4}Organismos}
\centering
\resizebox{\linewidth}{!}{
\begin{tabular}[t]{r|l|l|l|l|l|l|r|l|l}
\hiderowcolors
\hline
X1 & organismo & organismo\_descripcion & email\_oficial & fuero & circunscripcion & localidad & categoria & tipo & materia\\
\hline
\showrowcolors
1 & jdocco0000dia & Jdo Civ y Com Lab & jdocyc-dia@jusentrerios.gov.ar & Civil y Comercial & Diamante & Diamante & NA & jdo & cco|lab\\
\hline
2 & jdocco0000fed & Jdo Civ y Com Lab Fam & jdocyc-fcion@jusentrerios.gov.ar & Civil y Comercial & Federación & Federación & NA & jdo & cco|fam|lab\\
\hline
3 & jdocco0000frl & Jdo Civ y Com Lab & jdocyc-fral@jusentrerios.gov.ar & Civil y Comercial & Federal & Federal & NA & jdo & cco|lab\\
\hline
4 & jdocco0000ssa & Jdo Civ y Com Lab Fam & jdocyclab-ssdor@jusentrerios.gov.ar & Civil y Comercial & San Salvador & San Salvador & NA & jdo & cco|fam|lab\\
\hline
5 & jdocco0000tal & Jdo Civ y Com -ccomp.Laboral & jdocyc-tala@jusentrerios.gov.ar & Civil y Comercial & Tala & Rosario del Tala & NA & jdo & cco|lab\\
\hline
6 & jdocco0000vic & Jdo Civ y Com -ccomp.Laboral & jdocyc-vic@jusentrerios.gov.ar & Civil y Comercial & Victoria & Victoria & NA & jdo & cco|lab\\
\hline
\end{tabular}}
\end{table}

\rowcolors{2}{white}{white}

Se quitan los tipos de procesos \enquote{Monitorios}, ya que son de mero
trámite y no interesan en el análisis. Se eliminan 13687registros.

\begin{Shaded}
\begin{Highlighting}[]
\NormalTok{sentencias_1c <-}\StringTok{ }\NormalTok{sentencias_1c }\OperatorTok\StringTok{ }
\StringTok{  }\KeywordTok{filter}\NormalTok{(}\OperatorTok{!}\KeywordTok{grepl}\NormalTok{(}\StringTok{"MONITORIO"}\NormalTok{, tproc))}
\end{Highlighting}
\end{Shaded}

Se calcula duración como Fecha de Resolución menos Fecha de inicio, se
genera una nueva columna \enquote{duracion} y, en la siguiente gráfica,
se muestra un fragmento de la nueva tabla.

\rowcolors{2}{gray!6}{white}

\begin{table}

\caption{\label{tab:unnamed-chunk-6}Sentencias con duración}
\centering
\resizebox{\linewidth}{!}{
\begin{tabular}[t]{l|l|l|r|l|l|l|l|l|r|r|r|l|l}
\hiderowcolors
\hline
nro & tproc & as & ccon & finicio & fdesp & fvenc1 & fvenc2 & fres & tres & justiciables & reccap & iep & duracion\\
\hline
\showrowcolors
12858 & RESTRICCIONES A LA CAPACIDAD & S & 0 & 2015-09-15 & 28/11/2017 & 22/12/2017 & 19/02/2018 & 2017-12-29 & 7 & 1 & 0 & jdofam0002gch & 836 days\\
\hline
11852/5 & INCIDENTE & S & 1 & 2015-04-15 & 19/09/2017 & 04/10/2017 & 19/10/2017 & 2017-12-29 & 6 & 3 & 0 & jdofam0002gch & 989 days\\
\hline
12237 & ORDINARIO FILIACION E INDEMNIZACION DE DAÑOS & S & 1 & 2014-12-03 & 12/10/2017 & 13/12/2017 & 19/03/2018 & 2017-12-29 & 7 & 1 & 0 & jdofam0002gch & 1122 days\\
\hline
14440 & MEDIDA CAUTELAR (FAMILIA) & S & 0 & 2017-04-21 & 29/11/2017 & 18/12/2017 & 02/02/2018 & 2017-12-29 & 7 & 1 & 0 & jdofam0002gch & 252 days\\
\hline
11507 & ORDINARIO DAÑOS Y PERJUICIOS & S & 1 & 2014-03-13 & 30/11/2017 & 02/02/2018 & 06/04/2018 & 2017-12-29 & 7 & 2 & 0 & jdofam0002gch & 1387 days\\
\hline
8133 & ORDINARIO FILIACION E INDEMNIZACION DE DAÑOS & S & 1 & 2010-06-17 & 16/03/2017 & 17/05/2017 & 28/07/2017 & 2017-12-29 & 7 & 2 & 0 & jdofam0002gch & 2752 days\\
\hline
\end{tabular}}
\end{table}

\rowcolors{2}{white}{white}

Se eliminan las filas que tienen datos inválidos de fecha (datos nulos o
futuros por error de tipeo). Se eliminan 16 registros.

\begin{Shaded}
\begin{Highlighting}[]
\NormalTok{sentencias_1c <-}\StringTok{ }\NormalTok{sentencias_1c }\OperatorTok\StringTok{ }
\StringTok{  }\KeywordTok{filter}\NormalTok{(}\OperatorTok{!}\KeywordTok{is.na}\NormalTok{(finicio)) }\OperatorTok
\StringTok{  }\KeywordTok{filter}\NormalTok{(}\OperatorTok{!}\KeywordTok{is.na}\NormalTok{(fres)) }\OperatorTok\StringTok{ }
\StringTok{  }\KeywordTok{filter}\NormalTok{(fres }\OperatorTok{<}\StringTok{ '2018-09-01'}\NormalTok{, finicio }\OperatorTok{<}\StringTok{ '2018-09-01'}\NormalTok{)}
\end{Highlighting}
\end{Shaded}

Se calculan los cuartiles 25\% y 75\% de duración por cada tipo de
proceso (tproc), y con estos parámetros se clasifican en rápido, normal
o demorado si duración se encuentra en cada uno de los rangos definidos.

\begin{itemize}
\tightlist
\item
  Rápido: duración menor al cuartil 25\%
\item
  Normal: duración entre el cuartil 25\% y el 75\%.
\item
  Demorado: duración mayor al cuartil 75\%.
\end{itemize}

\begin{Shaded}
\begin{Highlighting}[]
\NormalTok{demora <-}\StringTok{ }\NormalTok{sentencias_1c }\OperatorTok\StringTok{ }
\StringTok{  }\KeywordTok{group_by}\NormalTok{(tproc) }\OperatorTok\StringTok{ }
\StringTok{  }\KeywordTok{summarise}\NormalTok{(}\DataTypeTok{techo_rapido=}\KeywordTok{quantile}\NormalTok{(duracion, }\DataTypeTok{probs=}\FloatTok{0.25}\NormalTok{),}
            \DataTypeTok{piso_demorado=}\KeywordTok{quantile}\NormalTok{(duracion, }\DataTypeTok{probs=}\FloatTok{0.75}\NormalTok{))}

\NormalTok{sentencias_1c <-}\StringTok{ }\NormalTok{sentencias_1c }\OperatorTok\StringTok{ }
\StringTok{  }\KeywordTok{left_join}\NormalTok{(demora, }\DataTypeTok{by=}\StringTok{"tproc"}\NormalTok{) }\OperatorTok\StringTok{ }
\StringTok{  }\KeywordTok{mutate}\NormalTok{(}\DataTypeTok{rapido =}\NormalTok{ duracion }\OperatorTok{<=}\StringTok{ }\NormalTok{techo_rapido) }\OperatorTok\StringTok{ }
\StringTok{  }\KeywordTok{mutate}\NormalTok{(}\DataTypeTok{normal =}\NormalTok{ duracion }\OperatorTok{>}\StringTok{ }\NormalTok{techo_rapido }\OperatorTok{&}\StringTok{ }\NormalTok{duracion }
         \OperatorTok{<}\StringTok{ }\NormalTok{piso_demorado) }\OperatorTok\StringTok{ }
\StringTok{  }\KeywordTok{mutate}\NormalTok{(}\DataTypeTok{demorado =}\NormalTok{ duracion }\OperatorTok{>=}\StringTok{ }\NormalTok{piso_demorado) }\OperatorTok\StringTok{ }
\StringTok{  }\KeywordTok{select}\NormalTok{(}\OperatorTok{-}\NormalTok{duracion, }\OperatorTok{-}\NormalTok{techo_rapido, }\OperatorTok{-}\NormalTok{piso_demorado) }
\CommentTok{# quitando columnas temporales }
\end{Highlighting}
\end{Shaded}

En al siguiente tabla se muestra un fragmento de los datos con los
cambios realizados.

\rowcolors{2}{gray!6}{white}

\begin{table}

\caption{\label{tab:unnamed-chunk-9}Agregando columnas demora}
\centering
\resizebox{\linewidth}{!}{
\begin{tabular}[t]{l|l|l|r|l|l|l|l|l|r|r|r|l|l|l|l}
\hiderowcolors
\hline
nro & tproc & as & ccon & finicio & fdesp & fvenc1 & fvenc2 & fres & tres & justiciables & reccap & iep & rapido & normal & demorado\\
\hline
\showrowcolors
12858 & RESTRICCIONES A L... & S & 0 & 2015-09-15 & 28/11/2017 & 22/12/2017 & 19/02/2018 & 2017-12-29 & 7 & 1 & 0 & jdofam0002gch & FALSE & TRUE & FALSE\\
\hline
11852/5 & INCIDENTE & S & 1 & 2015-04-15 & 19/09/2017 & 04/10/2017 & 19/10/2017 & 2017-12-29 & 6 & 3 & 0 & jdofam0002gch & FALSE & FALSE & TRUE\\
\hline
12237 & ORDINARIO FILIACI... & S & 1 & 2014-12-03 & 12/10/2017 & 13/12/2017 & 19/03/2018 & 2017-12-29 & 7 & 1 & 0 & jdofam0002gch & FALSE & TRUE & FALSE\\
\hline
14440 & MEDIDA CAUTELAR (... & S & 0 & 2017-04-21 & 29/11/2017 & 18/12/2017 & 02/02/2018 & 2017-12-29 & 7 & 1 & 0 & jdofam0002gch & FALSE & FALSE & TRUE\\
\hline
11507 & ORDINARIO DAÑOS Y... & S & 1 & 2014-03-13 & 30/11/2017 & 02/02/2018 & 06/04/2018 & 2017-12-29 & 7 & 2 & 0 & jdofam0002gch & FALSE & TRUE & FALSE\\
\hline
8133 & ORDINARIO FILIACI... & S & 1 & 2010-06-17 & 16/03/2017 & 17/05/2017 & 28/07/2017 & 2017-12-29 & 7 & 2 & 0 & jdofam0002gch & FALSE & FALSE & TRUE\\
\hline
\end{tabular}}
\end{table}

\rowcolors{2}{white}{white}

Se agregan los datos de los organismos para tenerlos separados por
columna, actualmente el dato se encontraba en columna iep.

\begin{Shaded}
\begin{Highlighting}[]
\NormalTok{organismos <-}\StringTok{ }\NormalTok{organismos }\OperatorTok\StringTok{ }
\StringTok{  }\KeywordTok{select}\NormalTok{(organismo, circunscripcion, localidad, materia)}

\NormalTok{sentencias_1c <-}\StringTok{ }\NormalTok{sentencias_1c }\OperatorTok\StringTok{ }
\StringTok{  }\KeywordTok{left_join}\NormalTok{(organismos, }\DataTypeTok{by =} \KeywordTok{c}\NormalTok{(}\StringTok{'iep'}\NormalTok{=}\StringTok{'organismo'}\NormalTok{))}
\end{Highlighting}
\end{Shaded}

Se explora la variable capital reclamado, para definir los rangos y
categorizar, graficando un histograma del logaritmo.

\includegraphics{TPMineria_files/figure-latex/unnamed-chunk-11-1.pdf}

Se calculan los cuartiles para evaluar si sirven para parametrizar el
capital reclamado (reccap).

\begin{Shaded}
\begin{Highlighting}[]
\KeywordTok{print}\NormalTok{(}\StringTok{'1º Curtil:'}\NormalTok{)}
\end{Highlighting}
\end{Shaded}

\begin{verbatim}
## [1] "1º Curtil:"
\end{verbatim}

\begin{Shaded}
\begin{Highlighting}[]
\KeywordTok{quantile}\NormalTok{(}\KeywordTok{pull}\NormalTok{(sentencias_1c[,}\StringTok{'reccap'}\NormalTok{]),.}\DecValTok{25}\NormalTok{, }\DataTypeTok{na.rm =} \OtherTok{TRUE}\NormalTok{)}
\end{Highlighting}
\end{Shaded}

\begin{verbatim}
## 25% 
##   0
\end{verbatim}

\begin{Shaded}
\begin{Highlighting}[]
\KeywordTok{print}\NormalTok{(}\StringTok{'2º Curtil:'}\NormalTok{)}
\end{Highlighting}
\end{Shaded}

\begin{verbatim}
## [1] "2º Curtil:"
\end{verbatim}

\begin{Shaded}
\begin{Highlighting}[]
\KeywordTok{quantile}\NormalTok{(}\KeywordTok{pull}\NormalTok{(sentencias_1c[,}\StringTok{'reccap'}\NormalTok{]),.}\DecValTok{50}\NormalTok{, }\DataTypeTok{na.rm =} \OtherTok{TRUE}\NormalTok{)}
\end{Highlighting}
\end{Shaded}

\begin{verbatim}
## 50% 
##   0
\end{verbatim}

\begin{Shaded}
\begin{Highlighting}[]
\KeywordTok{print}\NormalTok{(}\StringTok{'3º Curtil:'}\NormalTok{)}
\end{Highlighting}
\end{Shaded}

\begin{verbatim}
## [1] "3º Curtil:"
\end{verbatim}

\begin{Shaded}
\begin{Highlighting}[]
\KeywordTok{quantile}\NormalTok{(}\KeywordTok{pull}\NormalTok{(sentencias_1c[,}\StringTok{'reccap'}\NormalTok{]),.}\DecValTok{75}\NormalTok{, }\DataTypeTok{na.rm =} \OtherTok{TRUE}\NormalTok{)}
\end{Highlighting}
\end{Shaded}

\begin{verbatim}
## 75% 
##   0
\end{verbatim}

\begin{Shaded}
\begin{Highlighting}[]
\CommentTok{#View(sentencias_1c)}
\end{Highlighting}
\end{Shaded}

Como los todos los cuartiles obtenidos son cero, se vuelven a calcular
los cuartiles sin los datos ceros.

\begin{Shaded}
\begin{Highlighting}[]
\CommentTok{#reccap_not_cero <- which(sentencias_1c$reccap != 0)}

\KeywordTok{print}\NormalTok{(}\StringTok{'1º Curtil:'}\NormalTok{)}
\end{Highlighting}
\end{Shaded}

\begin{verbatim}
## [1] "1º Curtil:"
\end{verbatim}

\begin{Shaded}
\begin{Highlighting}[]
\KeywordTok{quantile}\NormalTok{(}\KeywordTok{which}\NormalTok{(sentencias_1c}\OperatorTok{$}\NormalTok{reccap }\OperatorTok{!=}\StringTok{ }\DecValTok{0}\NormalTok{),.}\DecValTok{25}\NormalTok{)}
\end{Highlighting}
\end{Shaded}

\begin{verbatim}
##    25% 
## 3181.5
\end{verbatim}

\begin{Shaded}
\begin{Highlighting}[]
\KeywordTok{print}\NormalTok{(}\StringTok{'2º Curtil:'}\NormalTok{)}
\end{Highlighting}
\end{Shaded}

\begin{verbatim}
## [1] "2º Curtil:"
\end{verbatim}

\begin{Shaded}
\begin{Highlighting}[]
\KeywordTok{quantile}\NormalTok{(}\KeywordTok{which}\NormalTok{(sentencias_1c}\OperatorTok{$}\NormalTok{reccap }\OperatorTok{!=}\StringTok{ }\DecValTok{0}\NormalTok{),.}\DecValTok{50}\NormalTok{)}
\end{Highlighting}
\end{Shaded}

\begin{verbatim}
##    50% 
## 6271.5
\end{verbatim}

\begin{Shaded}
\begin{Highlighting}[]
\KeywordTok{print}\NormalTok{(}\StringTok{'3º Curtil:'}\NormalTok{)}
\end{Highlighting}
\end{Shaded}

\begin{verbatim}
## [1] "3º Curtil:"
\end{verbatim}

\begin{Shaded}
\begin{Highlighting}[]
\KeywordTok{quantile}\NormalTok{(}\KeywordTok{which}\NormalTok{(sentencias_1c}\OperatorTok{$}\NormalTok{reccap }\OperatorTok{!=}\StringTok{ }\DecValTok{0}\NormalTok{),.}\DecValTok{75}\NormalTok{)}
\end{Highlighting}
\end{Shaded}

\begin{verbatim}
##     75% 
## 8948.75
\end{verbatim}

\begin{Shaded}
\begin{Highlighting}[]
\NormalTok{capmedio <-}\StringTok{ }\KeywordTok{mean}\NormalTok{(}\KeywordTok{pull}\NormalTok{(sentencias_1c[,}\StringTok{'reccap'}\NormalTok{]))}

\NormalTok{sentencias_1c <-}\StringTok{ }\NormalTok{sentencias_1c }\OperatorTok\StringTok{ }
\StringTok{  }\KeywordTok{mutate}\NormalTok{(}\DataTypeTok{reccap_0 =}\NormalTok{ reccap }\OperatorTok{==}\StringTok{ }\DecValTok{0}\NormalTok{) }\OperatorTok\StringTok{ }
\StringTok{  }\KeywordTok{mutate}\NormalTok{(}\DataTypeTok{reccap_1 =}\NormalTok{ (reccap }\OperatorTok{<}\StringTok{ }\KeywordTok{quantile}\NormalTok{(}\KeywordTok{which}\NormalTok{(}
\NormalTok{    sentencias_1c}\OperatorTok{$}\NormalTok{reccap }\OperatorTok{!=}\StringTok{ }\DecValTok{0}\NormalTok{),.}\DecValTok{25}\NormalTok{)) }\OperatorTok{&}\StringTok{ }\NormalTok{(reccap}\OperatorTok{!=}\DecValTok{0}\NormalTok{)) }\OperatorTok\StringTok{ }
\StringTok{  }\KeywordTok{mutate}\NormalTok{(}\DataTypeTok{reccap_2 =}\NormalTok{ (reccap }\OperatorTok{>=}\StringTok{ }\KeywordTok{quantile}\NormalTok{(}\KeywordTok{which}\NormalTok{(}
\NormalTok{    sentencias_1c}\OperatorTok{$}\NormalTok{reccap }\OperatorTok{!=}\StringTok{ }\DecValTok{0}\NormalTok{),.}\DecValTok{25}\NormalTok{)) }\OperatorTok{&}\StringTok{ }\NormalTok{(reccap }\OperatorTok{<}\StringTok{ }\KeywordTok{quantile}\NormalTok{(}\KeywordTok{which}\NormalTok{(}
\NormalTok{      sentencias_1c}\OperatorTok{$}\NormalTok{reccap }\OperatorTok{!=}\StringTok{ }\DecValTok{0}\NormalTok{),.}\DecValTok{50}\NormalTok{))) }\OperatorTok\StringTok{ }
\StringTok{  }\KeywordTok{mutate}\NormalTok{(}\DataTypeTok{reccap_3 =}\NormalTok{ (reccap }\OperatorTok{>=}\StringTok{ }\KeywordTok{quantile}\NormalTok{(}\KeywordTok{which}\NormalTok{(}
\NormalTok{    sentencias_1c}\OperatorTok{$}\NormalTok{reccap }\OperatorTok{!=}\StringTok{ }\DecValTok{0}\NormalTok{),.}\DecValTok{50}\NormalTok{)) }\OperatorTok{&}\StringTok{ }\NormalTok{(reccap }\OperatorTok{<}\StringTok{ }\KeywordTok{quantile}\NormalTok{(}\KeywordTok{which}\NormalTok{(}
\NormalTok{      sentencias_1c}\OperatorTok{$}\NormalTok{reccap }\OperatorTok{!=}\StringTok{ }\DecValTok{0}\NormalTok{),.}\DecValTok{75}\NormalTok{))) }\OperatorTok\StringTok{ }
\StringTok{  }\KeywordTok{mutate}\NormalTok{(}\DataTypeTok{reccap_4 =}\NormalTok{ (reccap }\OperatorTok{>=}\StringTok{ }\KeywordTok{quantile}\NormalTok{(}\KeywordTok{which}\NormalTok{(}
\NormalTok{    sentencias_1c}\OperatorTok{$}\NormalTok{reccap }\OperatorTok{!=}\StringTok{ }\DecValTok{0}\NormalTok{),.}\DecValTok{75}\NormalTok{))) }
\end{Highlighting}
\end{Shaded}

A partir de los cuartiles obtenidos, se generan las siguientes
categorías:

\begin{itemize}
\tightlist
\item
  Capital reclamado igual a cero.
\item
  Capital reclamado distinto de cero y menor al cuartil 25\%.
\item
  Capital reclamado entre los cuartiles 25\% y 50\%.
\item
  Capital reclamado entre los cuartiles 50\% y 75\%.
\item
  Capital reclamado mayor al cuartil 75\%.
\end{itemize}

Se separa la columna justiciables en los siguientes 6 rangos para
categorizar.

\begin{itemize}
\tightlist
\item
  Justiciables igual a 1
\item
  Justiciables igual a 2 o 3
\item
  Justiciables igual a 4 o 5
\item
  Justiciables igual a 6 o 7
\item
  Justiciables igual a 8 o 9
\item
  Justiciables mayor a 9
\end{itemize}

Se expresan las variables localidad, tipo de proceso, circunscripción y
materia como factor, esto se requiere para aplicar apriori.

\begin{Shaded}
\begin{Highlighting}[]
\NormalTok{sentencias_1c <-}\StringTok{ }\NormalTok{sentencias_1c }\OperatorTok\StringTok{ }
\StringTok{  }\KeywordTok{mutate}\NormalTok{(}\DataTypeTok{localidad =} \KeywordTok{as.factor}\NormalTok{(localidad))}

\NormalTok{sentencias_1c <-}\StringTok{ }\NormalTok{sentencias_1c }\OperatorTok\StringTok{ }
\StringTok{  }\KeywordTok{mutate}\NormalTok{(}\DataTypeTok{tproc =} \KeywordTok{as.factor}\NormalTok{(tproc)) }\OperatorTok\StringTok{ }
\StringTok{  }\KeywordTok{mutate}\NormalTok{(}\DataTypeTok{circunscripcion =} \KeywordTok{as.factor}\NormalTok{(circunscripcion)) }\OperatorTok\StringTok{ }
\StringTok{  }\KeywordTok{mutate}\NormalTok{(}\DataTypeTok{materia =} \KeywordTok{as.factor}\NormalTok{(materia))}
\end{Highlighting}
\end{Shaded}

Generamos una nueva tabla con las columnas tipo booleanos y categóricas.
Se muestra un fragmento en la siguiente tabla.

\rowcolors{2}{gray!6}{white}

\begin{table}

\caption{\label{tab:unnamed-chunk-16}Tabla final a utilizar en el algoritmo apriori}
\centering
\resizebox{\linewidth}{!}{
\begin{tabular}[t]{l|l|l|l|l|l|l|l|l|l|l|l|l|l|l|l|l}
\hiderowcolors
\hline
tproc & rapido & normal & demorado & circunscripcion & materia & reccap\_0 & reccap\_1 & reccap\_2 & reccap\_3 & reccap\_4 & justiciables0\_1 & justiciables2\_3 & justiciables4\_5 & justiciables6\_7 & justiciables8\_9 & justiciables10\_N\\
\hline
\showrowcolors
RESTRICCIONES A LA CAPACIDAD & FALSE & TRUE & FALSE & Gualeguaychú & fam|pen & TRUE & FALSE & FALSE & FALSE & FALSE & TRUE & FALSE & FALSE & FALSE & FALSE & FALSE\\
\hline
INCIDENTE & FALSE & FALSE & TRUE & Gualeguaychú & fam|pen & TRUE & FALSE & FALSE & FALSE & FALSE & FALSE & TRUE & FALSE & FALSE & FALSE & FALSE\\
\hline
ORDINARIO FILIACION E INDEMNIZACION DE DAÑOS & FALSE & TRUE & FALSE & Gualeguaychú & fam|pen & TRUE & FALSE & FALSE & FALSE & FALSE & TRUE & FALSE & FALSE & FALSE & FALSE & FALSE\\
\hline
MEDIDA CAUTELAR (FAMILIA) & FALSE & FALSE & TRUE & Gualeguaychú & fam|pen & TRUE & FALSE & FALSE & FALSE & FALSE & TRUE & FALSE & FALSE & FALSE & FALSE & FALSE\\
\hline
ORDINARIO DAÑOS Y PERJUICIOS & FALSE & TRUE & FALSE & Gualeguaychú & fam|pen & TRUE & FALSE & FALSE & FALSE & FALSE & FALSE & TRUE & FALSE & FALSE & FALSE & FALSE\\
\hline
ORDINARIO FILIACION E INDEMNIZACION DE DAÑOS & FALSE & FALSE & TRUE & Gualeguaychú & fam|pen & TRUE & FALSE & FALSE & FALSE & FALSE & FALSE & TRUE & FALSE & FALSE & FALSE & FALSE\\
\hline
\end{tabular}}
\end{table}

\rowcolors{2}{white}{white}

\section{Generación de Reglas}\label{generacion-de-reglas}

Con los datos ya pre-procesados aplicamos apriori para generar las
reglas de asociación. Inicialmente tomamos como valores límite un
soporte de 0.001 y una confianza de 0.5.

\scriptsize

\begin{Shaded}
\begin{Highlighting}[]
\NormalTok{rules <-}\StringTok{ }\KeywordTok{apriori}\NormalTok{(sentencias_final, }\DataTypeTok{parameter =} \KeywordTok{list}\NormalTok{(}
  \DataTypeTok{supp=}\FloatTok{0.001}\NormalTok{, }\DataTypeTok{conf=}\FloatTok{0.5}\NormalTok{, }\DataTypeTok{minlen=}\DecValTok{2}\NormalTok{), }\DataTypeTok{appearance =} \KeywordTok{list}\NormalTok{(}
    \DataTypeTok{rhs=}\KeywordTok{c}\NormalTok{(}\StringTok{"demorado"}\NormalTok{, }\StringTok{"rapido"}\NormalTok{)))}
\end{Highlighting}
\end{Shaded}

\begin{verbatim}
## Apriori
## 
## Parameter specification:
##  confidence minval smax arem  aval originalSupport maxtime support minlen
##         0.5    0.1    1 none FALSE            TRUE       5   0.001      2
##  maxlen target   ext
##      10  rules FALSE
## 
## Algorithmic control:
##  filter tree heap memopt load sort verbose
##     0.1 TRUE TRUE  FALSE TRUE    2    TRUE
## 
## Absolute minimum support count: 11 
## 
## set item appearances ...[2 item(s)] done [0.00s].
## set transactions ...[275 item(s), 11576 transaction(s)] done [0.01s].
## sorting and recoding items ... [105 item(s)] done [0.00s].
## creating transaction tree ... done [0.01s].
## checking subsets of size 1 2 3 4 5 6 done [0.00s].
## writing ... [310 rule(s)] done [0.00s].
## creating S4 object  ... done [0.01s].
\end{verbatim}

\begin{Shaded}
\begin{Highlighting}[]
\KeywordTok{summary}\NormalTok{(rules)}
\end{Highlighting}
\end{Shaded}

\begin{verbatim}
## set of 310 rules
## 
## rule length distribution (lhs + rhs):sizes
##   2   3   4   5   6 
##   2  47 126 107  28 
## 
##    Min. 1st Qu.  Median    Mean 3rd Qu.    Max. 
##   2.000   4.000   4.000   4.361   5.000   6.000 
## 
## summary of quality measures:
##     support           confidence          lift           count      
##  Min.   :0.001037   Min.   :0.5000   Min.   :1.837   Min.   : 12.0  
##  1st Qu.:0.001296   1st Qu.:0.5385   1st Qu.:2.004   1st Qu.: 15.0  
##  Median :0.001814   Median :0.6197   Median :2.321   Median : 21.0  
##  Mean   :0.003352   Mean   :0.6674   Mean   :2.474   Mean   : 38.8  
##  3rd Qu.:0.003455   3rd Qu.:0.7582   3rd Qu.:2.840   3rd Qu.: 40.0  
##  Max.   :0.014599   Max.   :1.0000   Max.   :3.749   Max.   :169.0  
## 
## mining info:
##              data ntransactions support confidence
##  sentencias_final         11576   0.001        0.5
\end{verbatim}

\normalsize

Graficamos las reglas para ver como varia el soporte y la confianza.

\includegraphics{TPMineria_files/figure-latex/unnamed-chunk-18-1.pdf} Se
grafica nuevamente pero incluyendo en dato del orden de las reglas con
colores.

\begin{verbatim}
## To reduce overplotting, jitter is added! Use jitter = 0 to prevent jitter.
\end{verbatim}

\includegraphics{TPMineria_files/figure-latex/unnamed-chunk-19-1.pdf}

Se realiza un inspect de las primeras reglas. Se puede ver que hay
reglas que no son de interés por tener baja confianza.

\scriptsize

\begin{Shaded}
\begin{Highlighting}[]
\KeywordTok{inspect}\NormalTok{(rules[}\DecValTok{1}\OperatorTok{:}\DecValTok{8}\NormalTok{])}
\end{Highlighting}
\end{Shaded}

\begin{verbatim}
##     lhs                                          rhs            support confidence     lift count
## [1] {circunscripcion=San Salvador}            => {rapido}   0.002505183  0.5000000 1.836877    29
## [2] {justiciables10_N}                        => {demorado} 0.004060124  0.5164835 1.936144    47
## [3] {tproc=EJECUTIVO,                                                                            
##      circunscripcion=San Salvador}            => {rapido}   0.001295784  0.5555556 2.040975    15
## [4] {circunscripcion=San Salvador,                                                               
##      materia=paz}                             => {rapido}   0.001814098  0.6176471 2.269084    21
## [5] {circunscripcion=San Salvador,                                                               
##      reccap_0}                                => {rapido}   0.002418798  0.5714286 2.099288    28
## [6] {circunscripcion=San Salvador,                                                               
##      justiciables2_3}                         => {rapido}   0.002159641  0.5555556 2.040975    25
## [7] {tproc=DIVORCIO POR MUTUO CONSENTIMIENTO,                                                    
##      circunscripcion=Gualeguay}               => {rapido}   0.001641327  0.9047619 3.323873    19
## [8] {materia=cco,                                                                                
##      justiciables10_N}                        => {demorado} 0.002764340  0.5000000 1.874352    32
\end{verbatim}

\normalsize

Se elimianan las reglas redundantes y se imprimen las primeras 8. Se
puede observar que hay reglas son similares pero solamente difieren en
su nivel especificidad, por esto deben ser eliminadas para el análisis.
\scriptsize

\begin{Shaded}
\begin{Highlighting}[]
\NormalTok{rules <-}\StringTok{ }\NormalTok{rules[}\OperatorTok{!}\KeywordTok{is.redundant}\NormalTok{(rules)]}
\KeywordTok{inspect}\NormalTok{(rules[}\DecValTok{1}\OperatorTok{:}\DecValTok{8}\NormalTok{])}
\end{Highlighting}
\end{Shaded}

\begin{verbatim}
##     lhs                                          rhs            support confidence     lift count
## [1] {circunscripcion=San Salvador}            => {rapido}   0.002505183  0.5000000 1.836877    29
## [2] {justiciables10_N}                        => {demorado} 0.004060124  0.5164835 1.936144    47
## [3] {tproc=EJECUTIVO,                                                                            
##      circunscripcion=San Salvador}            => {rapido}   0.001295784  0.5555556 2.040975    15
## [4] {circunscripcion=San Salvador,                                                               
##      materia=paz}                             => {rapido}   0.001814098  0.6176471 2.269084    21
## [5] {circunscripcion=San Salvador,                                                               
##      reccap_0}                                => {rapido}   0.002418798  0.5714286 2.099288    28
## [6] {circunscripcion=San Salvador,                                                               
##      justiciables2_3}                         => {rapido}   0.002159641  0.5555556 2.040975    25
## [7] {tproc=DIVORCIO POR MUTUO CONSENTIMIENTO,                                                    
##      circunscripcion=Gualeguay}               => {rapido}   0.001641327  0.9047619 3.323873    19
## [8] {circunscripcion=Paraná,                                                                     
##      justiciables10_N}                        => {demorado} 0.003023497  0.5384615 2.018533    35
\end{verbatim}

\normalsize

Se vuelven a visualizar las reglas pero ordenas por soporte y confianza.

\scriptsize

\begin{Shaded}
\begin{Highlighting}[]
\NormalTok{top.confidence <-}\StringTok{ }\KeywordTok{sort}\NormalTok{(rules, }\DataTypeTok{decreasing =} \OtherTok{TRUE}\NormalTok{, }
                       \DataTypeTok{na.last =} \OtherTok{NA}\NormalTok{, }\DataTypeTok{by =} \StringTok{"confidence"}\NormalTok{)}
\KeywordTok{inspect}\NormalTok{(top.confidence[}\DecValTok{1}\OperatorTok{:}\DecValTok{8}\NormalTok{])}
\end{Highlighting}
\end{Shaded}

\begin{verbatim}
##     lhs                                rhs            support confidence     lift count
## [1] {tproc=ACCION DE AMPARO,                                                           
##      circunscripcion=Nogoyá,                                                           
##      justiciables0_1}               => {demorado} 0.001382170  1.0000000 3.748705    16
## [2] {tproc=EJECUTIVO,                                                                  
##      circunscripcion=Paraná,                                                           
##      reccap_1,                                                                         
##      justiciables0_1}               => {rapido}   0.001295784  1.0000000 3.673754    15
## [3] {tproc=ACCION DE AMPARO,                                                           
##      circunscripcion=Nogoyá,                                                           
##      materia=paz}                   => {demorado} 0.002591569  0.9677419 3.627779    30
## [4] {tproc=DIVORCIO,                                                                   
##      circunscripcion=Gualeguay}     => {rapido}   0.002332412  0.9642857 3.542549    27
## [5] {tproc=EJECUTIVO,                                                                  
##      circunscripcion=Paraná,                                                           
##      justiciables0_1}               => {rapido}   0.001727713  0.9523810 3.498814    20
## [6] {circunscripcion=Paraná,                                                           
##      reccap_1,                                                                         
##      justiciables0_1}               => {rapido}   0.001382170  0.9411765 3.457651    16
## [7] {tproc=EJECUTIVO,                                                                  
##      reccap_1,                                                                         
##      justiciables0_1}               => {rapido}   0.001295784  0.9375000 3.444145    15
## [8] {tproc=EJECUCION DE HONORARIOS,                                                    
##      circunscripcion=Paraná,                                                           
##      materia=cco,                                                                      
##      justiciables0_1}               => {rapido}   0.001209399  0.9333333 3.428837    14
\end{verbatim}

\normalsize

\scriptsize

\begin{Shaded}
\begin{Highlighting}[]
\NormalTok{top.support <-}\StringTok{ }\KeywordTok{sort}\NormalTok{(rules, }\DataTypeTok{decreasing =} \OtherTok{TRUE}\NormalTok{, }
                    \DataTypeTok{na.last =} \OtherTok{NA}\NormalTok{, }\DataTypeTok{by =} \StringTok{"support"}\NormalTok{)}
\KeywordTok{inspect}\NormalTok{(top.support[}\DecValTok{1}\OperatorTok{:}\DecValTok{8}\NormalTok{])}
\end{Highlighting}
\end{Shaded}

\begin{verbatim}
##     lhs                               rhs           support confidence     lift count
## [1] {tproc=APREMIO,                                                                  
##      reccap_1}                     => {demorado} 0.01459917  0.5577558 2.090862   169
## [2] {tproc=APREMIO,                                                                  
##      reccap_1,                                                                       
##      justiciables2_3}              => {demorado} 0.01451279  0.5600000 2.099275   168
## [3] {tproc=APREMIO,                                                                  
##      materia=paz,                                                                    
##      reccap_1}                     => {demorado} 0.01442640  0.5585284 2.093758   167
## [4] {tproc=APREMIO,                                                                  
##      materia=paz,                                                                    
##      reccap_1,                                                                       
##      justiciables2_3}              => {demorado} 0.01434001  0.5608108 2.102314   166
## [5] {tproc=APREMIO,                                                                  
##      reccap_0}                     => {rapido}   0.01416724  0.6007326 2.206944   164
## [6] {circunscripcion=Uruguay,                                                        
##      justiciables2_3}              => {demorado} 0.01338977  0.5000000 1.874352   155
## [7] {tproc=APREMIO,                                                                  
##      reccap_0,                                                                       
##      justiciables2_3}              => {rapido}   0.01321700  0.6023622 2.212931   153
## [8] {tproc=EJECUTIVO,                                                                
##      circunscripcion=Gualeguaychú} => {demorado} 0.01278507  0.5362319 2.010175   148
\end{verbatim}

\normalsize

Se toman las reglas ordenandas por confianza para analizar.

\scriptsize

\begin{Shaded}
\begin{Highlighting}[]
\KeywordTok{summary}\NormalTok{(top.confidence)}
\end{Highlighting}
\end{Shaded}

\begin{verbatim}
## set of 140 rules
## 
## rule length distribution (lhs + rhs):sizes
##  2  3  4  5  6 
##  2 46 70 21  1 
## 
##    Min. 1st Qu.  Median    Mean 3rd Qu.    Max. 
##   2.000   3.000   4.000   3.807   4.000   6.000 
## 
## summary of quality measures:
##     support           confidence          lift           count       
##  Min.   :0.001037   Min.   :0.5000   Min.   :1.837   Min.   : 12.00  
##  1st Qu.:0.001296   1st Qu.:0.5351   1st Qu.:1.988   1st Qu.: 15.00  
##  Median :0.001814   Median :0.5888   Median :2.204   Median : 21.00  
##  Mean   :0.003960   Mean   :0.6396   Mean   :2.369   Mean   : 45.84  
##  3rd Qu.:0.004730   3rd Qu.:0.7103   3rd Qu.:2.657   3rd Qu.: 54.75  
##  Max.   :0.014599   Max.   :1.0000   Max.   :3.749   Max.   :169.00  
## 
## mining info:
##              data ntransactions support confidence
##  sentencias_final         11576   0.001        0.5
\end{verbatim}

\begin{Shaded}
\begin{Highlighting}[]
\KeywordTok{inspect}\NormalTok{(top.confidence[}\DecValTok{1}\OperatorTok{:}\DecValTok{8}\NormalTok{])}
\end{Highlighting}
\end{Shaded}

\begin{verbatim}
##     lhs                                rhs            support confidence     lift count
## [1] {tproc=ACCION DE AMPARO,                                                           
##      circunscripcion=Nogoyá,                                                           
##      justiciables0_1}               => {demorado} 0.001382170  1.0000000 3.748705    16
## [2] {tproc=EJECUTIVO,                                                                  
##      circunscripcion=Paraná,                                                           
##      reccap_1,                                                                         
##      justiciables0_1}               => {rapido}   0.001295784  1.0000000 3.673754    15
## [3] {tproc=ACCION DE AMPARO,                                                           
##      circunscripcion=Nogoyá,                                                           
##      materia=paz}                   => {demorado} 0.002591569  0.9677419 3.627779    30
## [4] {tproc=DIVORCIO,                                                                   
##      circunscripcion=Gualeguay}     => {rapido}   0.002332412  0.9642857 3.542549    27
## [5] {tproc=EJECUTIVO,                                                                  
##      circunscripcion=Paraná,                                                           
##      justiciables0_1}               => {rapido}   0.001727713  0.9523810 3.498814    20
## [6] {circunscripcion=Paraná,                                                           
##      reccap_1,                                                                         
##      justiciables0_1}               => {rapido}   0.001382170  0.9411765 3.457651    16
## [7] {tproc=EJECUTIVO,                                                                  
##      reccap_1,                                                                         
##      justiciables0_1}               => {rapido}   0.001295784  0.9375000 3.444145    15
## [8] {tproc=EJECUCION DE HONORARIOS,                                                    
##      circunscripcion=Paraná,                                                           
##      materia=cco,                                                                      
##      justiciables0_1}               => {rapido}   0.001209399  0.9333333 3.428837    14
\end{verbatim}

\normalsize

\section{Resultados / Discusión}\label{resultados-discusion}

\subsection{Asociaciones destacadas:}\label{asociaciones-destacadas}

\begin{itemize}
\tightlist
\item
  tipo de proceso \emph{ACCIÓN DE AMPARO} en las circunscripción de
  \textbf{Nogoyá} como \emph{demoradas} con respecto a los valores
  provinciales.
\item
  \emph{APREMIO} con capital reclamado \emph{recap\_1} aparecen en gral
  \emph{demorados}, sin embargo, cuando el capital reclamado es
  \emph{recap\_0}, se resuelven \emph{rápido}.
\item
  Los procesos \emph{EJECUTIVO} en \emph{Gualeguaychú} aparecen como
  \emph{demorados}.
\end{itemize}

Han aparecido reglas que eran de esperarse debido, por ejemplo, dado un
tipo de proceso, al incrementarse el capital reclamado o el nro de
justiciables, se puede inferir que todo el proceso se hace más complejo
y se elongue en el tiempo su resolución.

Sin embargo, \emph{no hay razones} de tipo procesal, para que
\emph{diferentes circunscrupciones/jurisdicciones presenten
diferencias}, las mismas, \emph{son las más significativas para el
análisis} y requieren un análisis más profundo. Las mismas pueden poner
en evidencia diferencias en dotaciones de personal, prácticas
administrativas y/o alguna circunstancia particular que lleve a estas
diferencias.

\section{Conclusiones}\label{conclusiones}

Las técnicas empleadas aquí han dado como resultado reglas interesantes
para investigar, ya que a priori no se esperaban circunscripciones
asociadas a diferencias en tiempos de resolución de sentencia, dichas
diferencias pueden estar asociadas al modo de trabajo en esas
localidades, quizás a la dotación de personal, capacitaciones de los
mismos, u otras razones que requieren investigación específica.

En cuanto a las herramientas, el preprocesamiento de datos y la
generación de reglas han resultado muy sencillo con las técnicas
utilizadas, así mismo el formato seleccionado para la realización del
informe, permite hacer evaluaciones interactivas mientras se conforma el
documento, como así también hace muy sencillo trabajar de manera
colaborativa ya que en el mismo documento está el código que se ha
utilizado para manipular los datos y generar las reglas.

\section{Referencias}\label{referencias}

\ldots{}
\newpage

\printbibliography

\end{document}
