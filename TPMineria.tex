\RequirePackage{fixltx2e}
\documentclass[runningheads,a4paper]{llncs}

\usepackage[american]{babel}
\usepackage{booktabs}
\usepackage{longtable}
\usepackage{array}
\usepackage{multirow}
\usepackage[table]{xcolor}
\usepackage{wrapfig}
\usepackage{float}
\usepackage{colortbl}
\usepackage{pdflscape}
\usepackage{tabu}
\usepackage{threeparttable}
\usepackage{threeparttablex}
\usepackage[normalem]{ulem}
\usepackage{makecell}
\usepackage{graphicx}

%extended enumerate, such as \begin{compactenum}
\usepackage{paralist}

%put figures inside a text
%\usepackage{picins}
%use
%\piccaptioninside
%\piccaption{...}
%\parpic[r]{\includegraphics ...}
%Text...

%Sorts the citations in the brackets
%\usepackage{cite}

%for easy quotations: \enquote{text}
\usepackage{csquotes}

\usepackage[T1]{fontenc}

%better font, similar to the default springer font
\usepackage{lmodern}
%if more space is needed, exchange lmodern by mathptmx
%\usepackage{mathptmx}

%enable margin kerning
\usepackage{microtype}

%for demonstration purposes only
\usepackage[math]{blindtext}

\usepackage{ifxetex,ifluatex}
\ifxetex
  \usepackage{fontspec,xltxtra,xunicode}
  \defaultfontfeatures{Mapping=tex-text,Scale=MatchLowercase}
  \newcommand{\euro}{€}
\else
  \ifluatex
    \usepackage{fontspec}
    \defaultfontfeatures{Mapping=tex-text,Scale=MatchLowercase}
    \newcommand{\euro}{€}
  \else
    \usepackage[utf8]{inputenc}
    \usepackage{eurosym}
  \fi
\fi


\makeatletter
\renewcommand\subsubsection{\@startsection{subsubsection}{3}{\z@}%
                       {-18\p@ \@plus -4\p@ \@minus -4\p@}%
                       {4\p@ \@plus 2\p@ \@minus 2\p@}%
                       {\normalfont\normalsize\bfseries\boldmath
                        \rightskip=\z@ \@plus 8em\pretolerance=10000 }}
\renewcommand\paragraph{\@startsection{paragraph}{4}{\z@}%
                       {-12\p@ \@plus -4\p@ \@minus -4\p@}%
                       {2\p@ \@plus 1\p@ \@minus 1\p@}%
                       {\normalfont\normalsize\itshape
                        \rightskip=\z@ \@plus 8em\pretolerance=10000 }}
\makeatother


%\usepackage[capitalise,nameinlink]{cleveref}
%Nice formats for \cref
%\crefname{section}{Sect.}{Sect.}
%\Crefname{section}{Section}{Sections}
%\crefname{figure}{Fig.}{Fig.}
%\Crefname{figure}{Figure}{Figures}

\usepackage{xspace}
%\newcommand{\eg}{e.\,g.\xspace}
%\newcommand{\ie}{i.\,e.\xspace}
\newcommand{\eg}{e.\,g.,\ }
\newcommand{\ie}{i.\,e.,\ }

% correct bad hyphenation here
\hyphenation{op-tical net-works semi-conduc-tor}

%%%%%%%%%%%%%%%%%%%%%%%%%%%%%%%%%%%%%%%%%%%%%%%%
\usepackage{fancyvrb}
%\DefineShortVerb[commandchars=\\\{\}]{\|}
\DefineVerbatimEnvironment{Highlighting}{Verbatim}{commandchars=\\\{\}}
% Add ',fontsize=\small' for more characters per line
\newenvironment{Shaded}{}{}
\newcommand{\KeywordTok}[1]{\textcolor[rgb]{0.00,0.44,0.13}{\textbf{{#1}}}}
\newcommand{\DataTypeTok}[1]{\textcolor[rgb]{0.56,0.13,0.00}{{#1}}}
\newcommand{\DecValTok}[1]{\textcolor[rgb]{0.25,0.63,0.44}{{#1}}}
\newcommand{\BaseNTok}[1]{\textcolor[rgb]{0.25,0.63,0.44}{{#1}}}
\newcommand{\FloatTok}[1]{\textcolor[rgb]{0.25,0.63,0.44}{{#1}}}
\newcommand{\CharTok}[1]{\textcolor[rgb]{0.25,0.44,0.63}{{#1}}}
\newcommand{\StringTok}[1]{\textcolor[rgb]{0.25,0.44,0.63}{{#1}}}
\newcommand{\CommentTok}[1]{\textcolor[rgb]{0.38,0.63,0.69}{\textit{{#1}}}}
\newcommand{\OtherTok}[1]{\textcolor[rgb]{0.00,0.44,0.13}{{#1}}}
\newcommand{\AlertTok}[1]{\textcolor[rgb]{1.00,0.00,0.00}{\textbf{{#1}}}}
\newcommand{\FunctionTok}[1]{\textcolor[rgb]{0.02,0.16,0.49}{{#1}}}
\newcommand{\RegionMarkerTok}[1]{{#1}}
\newcommand{\ErrorTok}[1]{\textcolor[rgb]{1.00,0.00,0.00}{\textbf{{#1}}}}
\newcommand{\NormalTok}[1]{{#1}}
\newcommand{\ConstantTok}[1]{\textcolor[rgb]{0.53,0.00,0.00}{{#1}}}
\newcommand{\SpecialCharTok}[1]{\textcolor[rgb]{0.25,0.44,0.63}{{#1}}}
\newcommand{\VerbatimStringTok}[1]{\textcolor[rgb]{0.25,0.44,0.63}{{#1}}}
\newcommand{\SpecialStringTok}[1]{\textcolor[rgb]{0.73,0.40,0.53}{{#1}}}
\newcommand{\ImportTok}[1]{{#1}}
\newcommand{\DocumentationTok}[1]{\textcolor[rgb]{0.73,0.13,0.13}{\textit{{#1}}}}
\newcommand{\AnnotationTok}[1]{\textcolor[rgb]{0.38,0.63,0.69}{\textbf{\textit{{#1}}}}}
\newcommand{\CommentVarTok}[1]{\textcolor[rgb]{0.38,0.63,0.69}{\textbf{\textit{{#1}}}}}
\newcommand{\VariableTok}[1]{\textcolor[rgb]{0.10,0.09,0.49}{{#1}}}
\newcommand{\ControlFlowTok}[1]{\textcolor[rgb]{0.00,0.44,0.13}{\textbf{{#1}}}}
\newcommand{\OperatorTok}[1]{\textcolor[rgb]{0.40,0.40,0.40}{{#1}}}
\newcommand{\BuiltInTok}[1]{{#1}}
\newcommand{\ExtensionTok}[1]{{#1}}
\newcommand{\PreprocessorTok}[1]{\textcolor[rgb]{0.74,0.48,0.00}{{#1}}}
\newcommand{\AttributeTok}[1]{\textcolor[rgb]{0.49,0.56,0.16}{{#1}}}
\newcommand{\InformationTok}[1]{\textcolor[rgb]{0.38,0.63,0.69}{\textbf{\textit{{#1}}}}}
\newcommand{\WarningTok}[1]{\textcolor[rgb]{0.38,0.63,0.69}{\textbf{\textit{{#1}}}}}
\ifxetex
  \usepackage[setpagesize=false, % page size defined by xetex
              unicode=false, % unicode breaks when used with xetex
              xetex,
              colorlinks=true,
              linkcolor=blue]{hyperref}
\else
%  \usepackage[unicode=true,
%              colorlinks=true,
%              linkcolor=blue]{hyperref}
    %unobstrusive usage of hyperref
    \ifnum\pdfoutput>0
        \usepackage[
        unicode=true,
        %pdfauthor={},
        %pdfsubject={},
        %pdftitle={},
        %pdfkeywords={},
        bookmarks=false,
        breaklinks=true,
        colorlinks=true,
        linkcolor=black,
        citecolor=black,
        urlcolor=black,
        %pdfstartpage=19,
        pdfpagelayout=SinglePage
        ]{hyperref}
        %enables correct jumping to figures when referencing
        \usepackage[all]{hypcap}
    \else
        \usepackage{hyperref}
    \fi
\fi
\hypersetup{breaklinks=true, pdfborder={0 0 0}}
\setlength{\parindent}{15pt} % set to 0pt if you want no indent in the first line of a paragraph
%\setlength{\parskip}{6pt plus 2pt minus 1pt}
\setlength{\parskip}{0pt}
\setlength{\emergencystretch}{3em}  % prevent overfull lines
\setcounter{secnumdepth}{3}
%\EndDefineVerbatimEnvironment{Highlighting}

\raggedbottom % allow ragged page bottoms

\def\tightlist{} % fix error when translating md lists to itemize

%%%%%%%%%%%%%%%%%%%%%%%%%%%%%%%%%%%%%%%%%%%%%%%%

\begin{document}


\title{TPMineria}
%If Title is too long, use \titlerunning
%\titlerunning{Short Title}
%Single insitute
\author{Zacarías F. Ojeda}

%If there are too many authors, use \authorrunning
%\authorrunning{First Author et al.}


\institute{
}

%Multiple insitutes
%Currently disabled
%
\iffalse
\authorinfo{
  
}{
}{
  \{\}
}
\fi


\maketitle



Cargamos los datasets originales

\begin{Shaded}
\begin{Highlighting}[]
\NormalTok{sentencias_1c <-}\StringTok{ }\KeywordTok{read_csv}\NormalTok{(}\StringTok{"./sentencias_1c.csv"}\NormalTok{) }\OperatorTok\StringTok{ }
\StringTok{  }\KeywordTok{filter}\NormalTok{(}\KeywordTok{is.na}\NormalTok{(mat) }\OperatorTok{|}\StringTok{ }\KeywordTok{toupper}\NormalTok{(mat)}\OperatorTok{==}\StringTok{"C"}\NormalTok{) }\OperatorTok\StringTok{ }\CommentTok{# solo incluir materia civil  }
\StringTok{  }\KeywordTok{select}\NormalTok{(}\OperatorTok{-}\NormalTok{mat) }

\NormalTok{organismos <-}\StringTok{ }\KeywordTok{read_csv}\NormalTok{(}\StringTok{"./organismos.csv"}\NormalTok{)}
\end{Highlighting}
\end{Shaded}

Inspección de los datos

\begin{Shaded}
\begin{Highlighting}[]
\NormalTok{sentencias_1c }\OperatorTok\StringTok{ }
\StringTok{  }\KeywordTok{mostrar}\NormalTok{(}\DataTypeTok{caption =} \StringTok{"Sentencias Primera Instancia Original"}\NormalTok{)}
\end{Highlighting}
\end{Shaded}

\rowcolors{2}{gray!6}{white}

\begin{table}

\caption{\label{tab:unnamed-chunk-3}Sentencias Primera Instancia Original}
\centering
\resizebox{\linewidth}{!}{
\begin{tabular}[t]{l|l|l|r|l|l|l|l|l|r|r|r|l}
\hiderowcolors
\hline
nro & tproc & as & ccon & finicio & fdesp & fvenc1 & fvenc2 & fres & tres & justiciables & reccap & iep\\
\hline
\showrowcolors
12858 & RESTRICCIONES A LA CAPACIDAD & S & 0 & 15/09/2015 & 28/11/2017 & 22/12/2017 & 19/02/2018 & 29/12/2017 & 7 & 1 & 0 & jdofam0002gch\\
\hline
11852/5 & INCIDENTE & S & 1 & 15/04/2015 & 19/09/2017 & 04/10/2017 & 19/10/2017 & 29/12/2017 & 6 & 3 & 0 & jdofam0002gch\\
\hline
12237 & ORDINARIO FILIACION E INDEMNIZACION DE DAÑOS & S & 1 & 03/12/2014 & 12/10/2017 & 13/12/2017 & 19/03/2018 & 29/12/2017 & 7 & 1 & 0 & jdofam0002gch\\
\hline
14440 & MEDIDA CAUTELAR (FAMILIA) & S & 0 & 21/04/2017 & 29/11/2017 & 18/12/2017 & 02/02/2018 & 29/12/2017 & 7 & 1 & 0 & jdofam0002gch\\
\hline
11507 & ORDINARIO DAÑOS Y PERJUICIOS & S & 1 & 13/03/2014 & 30/11/2017 & 02/02/2018 & 06/04/2018 & 29/12/2017 & 7 & 2 & 0 & jdofam0002gch\\
\hline
8133 & ORDINARIO FILIACION E INDEMNIZACION DE DAÑOS & S & 1 & 17/06/2010 & 16/03/2017 & 17/05/2017 & 28/07/2017 & 29/12/2017 & 7 & 2 & 0 & jdofam0002gch\\
\hline
\end{tabular}}
\end{table}

\rowcolors{2}{white}{white}

\begin{Shaded}
\begin{Highlighting}[]
\NormalTok{organismos }\OperatorTok\StringTok{ }
\StringTok{  }\KeywordTok{mostrar}\NormalTok{(}\DataTypeTok{caption =} \StringTok{"Organismos"}\NormalTok{)}
\end{Highlighting}
\end{Shaded}

\rowcolors{2}{gray!6}{white}

\begin{table}

\caption{\label{tab:unnamed-chunk-4}Organismos}
\centering
\resizebox{\linewidth}{!}{
\begin{tabular}[t]{r|l|l|l|l|l|l|r|l|l}
\hiderowcolors
\hline
X1 & organismo & organismo\_descripcion & email\_oficial & fuero & circunscripcion & localidad & categoria & tipo & materia\\
\hline
\showrowcolors
1 & jdocco0000dia & Jdo Civ y Com Lab & jdocyc-dia@jusentrerios.gov.ar & Civil y Comercial & Diamante & Diamante & NA & jdo & cco|lab\\
\hline
2 & jdocco0000fed & Jdo Civ y Com Lab Fam & jdocyc-fcion@jusentrerios.gov.ar & Civil y Comercial & Federación & Federación & NA & jdo & cco|fam|lab\\
\hline
3 & jdocco0000frl & Jdo Civ y Com Lab & jdocyc-fral@jusentrerios.gov.ar & Civil y Comercial & Federal & Federal & NA & jdo & cco|lab\\
\hline
4 & jdocco0000ssa & Jdo Civ y Com Lab Fam & jdocyclab-ssdor@jusentrerios.gov.ar & Civil y Comercial & San Salvador & San Salvador & NA & jdo & cco|fam|lab\\
\hline
5 & jdocco0000tal & Jdo Civ y Com -ccomp.Laboral & jdocyc-tala@jusentrerios.gov.ar & Civil y Comercial & Tala & Rosario del Tala & NA & jdo & cco|lab\\
\hline
6 & jdocco0000vic & Jdo Civ y Com -ccomp.Laboral & jdocyc-vic@jusentrerios.gov.ar & Civil y Comercial & Victoria & Victoria & NA & jdo & cco|lab\\
\hline
\end{tabular}}
\end{table}

\rowcolors{2}{white}{white}

Quitamos los tipos de procesos \enquote{Monitorios}, ya que son de mero
trámite y no interesan en el análisis

\begin{Shaded}
\begin{Highlighting}[]
\NormalTok{sentencias_1c <-}\StringTok{ }\NormalTok{sentencias_1c }\OperatorTok\StringTok{ }
\StringTok{  }\KeywordTok{filter}\NormalTok{(}\OperatorTok{!}\KeywordTok{grepl}\NormalTok{(}\StringTok{"MONITORIO"}\NormalTok{, tproc))}
\end{Highlighting}
\end{Shaded}

Calcula duracion como Fecha de Resolucion - Fecha de inicio

\begin{Shaded}
\begin{Highlighting}[]
\NormalTok{sentencias_1c <-}\StringTok{ }\NormalTok{sentencias_1c }\OperatorTok\StringTok{ }
\StringTok{  }\KeywordTok{mutate}\NormalTok{(}\DataTypeTok{finicio =}\NormalTok{ lubridate}\OperatorTok{::}\KeywordTok{dmy}\NormalTok{(finicio)) }\OperatorTok\StringTok{ }
\StringTok{  }\KeywordTok{mutate}\NormalTok{(}\DataTypeTok{fres =}\NormalTok{ lubridate}\OperatorTok{::}\KeywordTok{dmy}\NormalTok{(fres)) }\OperatorTok\StringTok{ }
\StringTok{  }\KeywordTok{mutate}\NormalTok{(}\DataTypeTok{duracion =}\NormalTok{ fres }\OperatorTok{-}\StringTok{ }\NormalTok{finicio)}

\NormalTok{sentencias_1c }\OperatorTok\StringTok{ }
\StringTok{  }\KeywordTok{mostrar}\NormalTok{(}\DataTypeTok{caption =} \StringTok{"Sentencias con duración")}
\end{Highlighting}
\end{Shaded}

\rowcolors{2}{gray!6}{white}

\begin{table}

\caption{\label{tab:unnamed-chunk-6}Sentencias con duración}
\centering
\resizebox{\linewidth}{!}{
\begin{tabular}[t]{l|l|l|r|l|l|l|l|l|r|r|r|l|l}
\hiderowcolors
\hline
nro & tproc & as & ccon & finicio & fdesp & fvenc1 & fvenc2 & fres & tres & justiciables & reccap & iep & duracion\\
\hline
\showrowcolors
12858 & RESTRICCIONES A LA CAPACIDAD & S & 0 & 2015-09-15 & 28/11/2017 & 22/12/2017 & 19/02/2018 & 2017-12-29 & 7 & 1 & 0 & jdofam0002gch & 836 days\\
\hline
11852/5 & INCIDENTE & S & 1 & 2015-04-15 & 19/09/2017 & 04/10/2017 & 19/10/2017 & 2017-12-29 & 6 & 3 & 0 & jdofam0002gch & 989 days\\
\hline
12237 & ORDINARIO FILIACION E INDEMNIZACION DE DAÑOS & S & 1 & 2014-12-03 & 12/10/2017 & 13/12/2017 & 19/03/2018 & 2017-12-29 & 7 & 1 & 0 & jdofam0002gch & 1122 days\\
\hline
14440 & MEDIDA CAUTELAR (FAMILIA) & S & 0 & 2017-04-21 & 29/11/2017 & 18/12/2017 & 02/02/2018 & 2017-12-29 & 7 & 1 & 0 & jdofam0002gch & 252 days\\
\hline
11507 & ORDINARIO DAÑOS Y PERJUICIOS & S & 1 & 2014-03-13 & 30/11/2017 & 02/02/2018 & 06/04/2018 & 2017-12-29 & 7 & 2 & 0 & jdofam0002gch & 1387 days\\
\hline
8133 & ORDINARIO FILIACION E INDEMNIZACION DE DAÑOS & S & 1 & 2010-06-17 & 16/03/2017 & 17/05/2017 & 28/07/2017 & 2017-12-29 & 7 & 2 & 0 & jdofam0002gch & 2752 days\\
\hline
\end{tabular}}
\end{table}

\rowcolors{2}{white}{white}

Eliminamos filas que tienen datos invalidos de fecha (datos nulos o
futuros por error de tipeo) Reemplazamos los datos NA de reccap por
cero.

\begin{Shaded}
\begin{Highlighting}[]
\NormalTok{sentencias_1c <-}\StringTok{ }\NormalTok{sentencias_1c }\OperatorTok\StringTok{ }
\StringTok{  }\KeywordTok{filter}\NormalTok{(}\OperatorTok{!}\KeywordTok{is.na}\NormalTok{(finicio)) }\OperatorTok
\StringTok{  }\KeywordTok{filter}\NormalTok{(}\OperatorTok{!}\KeywordTok{is.na}\NormalTok{(fres)) }\OperatorTok\StringTok{ }
\StringTok{  }\KeywordTok{filter}\NormalTok{(fres }\OperatorTok{<}\StringTok{ '2018-09-01'}\NormalTok{, finicio }\OperatorTok{<}\StringTok{ '2018-09-01'}\NormalTok{)}
\end{Highlighting}
\end{Shaded}

Calcula los cuartiles de duración por cada tipo de proceso (tproc), y se
clasifica en rapido / normal / demorado si duracion es mayor a media de
tproc utilizando one hot encoding.

\begin{Shaded}
\begin{Highlighting}[]
\NormalTok{demora <-}\StringTok{ }\NormalTok{sentencias_1c }\OperatorTok\StringTok{ }
\StringTok{  }\KeywordTok{group_by}\NormalTok{(tproc) }\OperatorTok\StringTok{ }
\StringTok{  }\KeywordTok{summarise}\NormalTok{(}\DataTypeTok{techo_rapido=}\KeywordTok{quantile}\NormalTok{(duracion, }\DataTypeTok{probs=}\FloatTok{0.25}\NormalTok{),}
            \DataTypeTok{piso_demorado=}\KeywordTok{quantile}\NormalTok{(duracion, }\DataTypeTok{probs=}\FloatTok{0.75}\NormalTok{))}

\NormalTok{sentencias_1c <-}\StringTok{ }\NormalTok{sentencias_1c }\OperatorTok\StringTok{ }
\StringTok{  }\KeywordTok{left_join}\NormalTok{(demora, }\DataTypeTok{by=}\StringTok{"tproc"}\NormalTok{) }\OperatorTok\StringTok{ }
\StringTok{  }\KeywordTok{mutate}\NormalTok{(}\DataTypeTok{rapido =}\NormalTok{ duracion }\OperatorTok{<=}\StringTok{ }\NormalTok{techo_rapido) }\OperatorTok\StringTok{ }
\StringTok{  }\KeywordTok{mutate}\NormalTok{(}\DataTypeTok{normal =}\NormalTok{ duracion }\OperatorTok{>}\StringTok{ }\NormalTok{techo_rapido }\OperatorTok{&}\StringTok{ }\NormalTok{duracion }\OperatorTok{<}\StringTok{ }\NormalTok{piso_demorado) }\OperatorTok\StringTok{ }
\StringTok{  }\KeywordTok{mutate}\NormalTok{(}\DataTypeTok{demorado =}\NormalTok{ duracion }\OperatorTok{>=}\StringTok{ }\NormalTok{piso_demorado) }\OperatorTok\StringTok{ }
\StringTok{  }\KeywordTok{select}\NormalTok{(}\OperatorTok{-}\NormalTok{duracion, }\OperatorTok{-}\NormalTok{techo_rapido, }\OperatorTok{-}\NormalTok{piso_demorado) }\CommentTok{# quitando columnas temporales }
\end{Highlighting}
\end{Shaded}

\begin{Shaded}
\begin{Highlighting}[]
\NormalTok{sentencias_1c }\OperatorTok\StringTok{ }
\StringTok{  }\KeywordTok{mutate}\NormalTok{(}\DataTypeTok{tproc =} \KeywordTok{str_trunc}\NormalTok{(tproc, }\DecValTok{20}\NormalTok{))}\OperatorTok\StringTok{ }
\StringTok{  }\KeywordTok{mostrar}\NormalTok{(}\DataTypeTok{caption =} \StringTok{"Agregando columnas demora"}\NormalTok{) }
\end{Highlighting}
\end{Shaded}

\rowcolors{2}{gray!6}{white}

\begin{table}

\caption{\label{tab:unnamed-chunk-9}Agregando columnas demora}
\centering
\resizebox{\linewidth}{!}{
\begin{tabular}[t]{l|l|l|r|l|l|l|l|l|r|r|r|l|l|l|l}
\hiderowcolors
\hline
nro & tproc & as & ccon & finicio & fdesp & fvenc1 & fvenc2 & fres & tres & justiciables & reccap & iep & rapido & normal & demorado\\
\hline
\showrowcolors
12858 & RESTRICCIONES A L... & S & 0 & 2015-09-15 & 28/11/2017 & 22/12/2017 & 19/02/2018 & 2017-12-29 & 7 & 1 & 0 & jdofam0002gch & FALSE & TRUE & FALSE\\
\hline
11852/5 & INCIDENTE & S & 1 & 2015-04-15 & 19/09/2017 & 04/10/2017 & 19/10/2017 & 2017-12-29 & 6 & 3 & 0 & jdofam0002gch & FALSE & FALSE & TRUE\\
\hline
12237 & ORDINARIO FILIACI... & S & 1 & 2014-12-03 & 12/10/2017 & 13/12/2017 & 19/03/2018 & 2017-12-29 & 7 & 1 & 0 & jdofam0002gch & FALSE & TRUE & FALSE\\
\hline
14440 & MEDIDA CAUTELAR (... & S & 0 & 2017-04-21 & 29/11/2017 & 18/12/2017 & 02/02/2018 & 2017-12-29 & 7 & 1 & 0 & jdofam0002gch & FALSE & FALSE & TRUE\\
\hline
11507 & ORDINARIO DAÑOS Y... & S & 1 & 2014-03-13 & 30/11/2017 & 02/02/2018 & 06/04/2018 & 2017-12-29 & 7 & 2 & 0 & jdofam0002gch & FALSE & TRUE & FALSE\\
\hline
8133 & ORDINARIO FILIACI... & S & 1 & 2010-06-17 & 16/03/2017 & 17/05/2017 & 28/07/2017 & 2017-12-29 & 7 & 2 & 0 & jdofam0002gch & FALSE & FALSE & TRUE\\
\hline
\end{tabular}}
\end{table}

\rowcolors{2}{white}{white}

Agrega datos de organismos para tenerlos separados por columna,
actualmente se encuentra en columna iep.

\begin{Shaded}
\begin{Highlighting}[]
\NormalTok{organismos <-}\StringTok{ }\NormalTok{organismos }\OperatorTok\StringTok{ }
\StringTok{  }\KeywordTok{select}\NormalTok{(organismo, circunscripcion, localidad, materia)}

\NormalTok{sentencias_1c <-}\StringTok{ }\NormalTok{sentencias_1c }\OperatorTok\StringTok{ }
\StringTok{  }\KeywordTok{left_join}\NormalTok{(organismos, }\DataTypeTok{by =} \KeywordTok{c}\NormalTok{(}\StringTok{'iep'}\NormalTok{=}\StringTok{'organismo'}\NormalTok{))}
\end{Highlighting}
\end{Shaded}

Exploremos la variable capital reclamado para definir los rangos

\begin{Shaded}
\begin{Highlighting}[]
\NormalTok{histograma <-}\StringTok{ }\NormalTok{sentencias_1c }\OperatorTok\StringTok{ }
\StringTok{  }\KeywordTok{ggplot}\NormalTok{() }\OperatorTok{+}
\StringTok{  }\KeywordTok{geom_histogram}\NormalTok{(}\KeywordTok{aes}\NormalTok{(}\DataTypeTok{x=}\KeywordTok{log}\NormalTok{(reccap)))}

\NormalTok{histograma}
\end{Highlighting}
\end{Shaded}

\begin{verbatim}
## `stat_bin()` using `bins = 30`. Pick better value with `binwidth`.
\end{verbatim}

\includegraphics{TPMineria_files/figure-latex/unnamed-chunk-11-1.pdf}

Calculamos los cuartiles para ver si nos sirven para parametrizar
(reccap)

\begin{Shaded}
\begin{Highlighting}[]
\KeywordTok{print}\NormalTok{(}\StringTok{'1º Curtil:'}\NormalTok{)}
\end{Highlighting}
\end{Shaded}

\begin{verbatim}
## [1] "1º Curtil:"
\end{verbatim}

\begin{Shaded}
\begin{Highlighting}[]
\KeywordTok{quantile}\NormalTok{(}\KeywordTok{pull}\NormalTok{(sentencias_1c[,}\StringTok{'reccap'}\NormalTok{]),.}\DecValTok{25}\NormalTok{, }\DataTypeTok{na.rm =} \OtherTok{TRUE}\NormalTok{)}
\end{Highlighting}
\end{Shaded}

\begin{verbatim}
## 25% 
##   0
\end{verbatim}

\begin{Shaded}
\begin{Highlighting}[]
\KeywordTok{print}\NormalTok{(}\StringTok{'2º Curtil:'}\NormalTok{)}
\end{Highlighting}
\end{Shaded}

\begin{verbatim}
## [1] "2º Curtil:"
\end{verbatim}

\begin{Shaded}
\begin{Highlighting}[]
\KeywordTok{quantile}\NormalTok{(}\KeywordTok{pull}\NormalTok{(sentencias_1c[,}\StringTok{'reccap'}\NormalTok{]),.}\DecValTok{50}\NormalTok{, }\DataTypeTok{na.rm =} \OtherTok{TRUE}\NormalTok{)}
\end{Highlighting}
\end{Shaded}

\begin{verbatim}
## 50% 
##   0
\end{verbatim}

\begin{Shaded}
\begin{Highlighting}[]
\KeywordTok{print}\NormalTok{(}\StringTok{'3º Curtil:'}\NormalTok{)}
\end{Highlighting}
\end{Shaded}

\begin{verbatim}
## [1] "3º Curtil:"
\end{verbatim}

\begin{Shaded}
\begin{Highlighting}[]
\KeywordTok{quantile}\NormalTok{(}\KeywordTok{pull}\NormalTok{(sentencias_1c[,}\StringTok{'reccap'}\NormalTok{]),.}\DecValTok{75}\NormalTok{, }\DataTypeTok{na.rm =} \OtherTok{TRUE}\NormalTok{)}
\end{Highlighting}
\end{Shaded}

\begin{verbatim}
## 75% 
##   0
\end{verbatim}

\begin{Shaded}
\begin{Highlighting}[]
\CommentTok{#View(sentencias_1c)}
\end{Highlighting}
\end{Shaded}

Como dos de los curtiles son cero, elimino los ceros y vuelvo a calcular
los cuartiles.

\begin{Shaded}
\begin{Highlighting}[]
\CommentTok{#reccap_not_cero <- which(sentencias_1c$reccap != 0)}

\KeywordTok{print}\NormalTok{(}\StringTok{'1º Curtil:'}\NormalTok{)}
\end{Highlighting}
\end{Shaded}

\begin{verbatim}
## [1] "1º Curtil:"
\end{verbatim}

\begin{Shaded}
\begin{Highlighting}[]
\KeywordTok{quantile}\NormalTok{(}\KeywordTok{which}\NormalTok{(sentencias_1c}\OperatorTok{$}\NormalTok{reccap }\OperatorTok{!=}\StringTok{ }\DecValTok{0}\NormalTok{),.}\DecValTok{25}\NormalTok{)}
\end{Highlighting}
\end{Shaded}

\begin{verbatim}
##    25% 
## 3181.5
\end{verbatim}

\begin{Shaded}
\begin{Highlighting}[]
\KeywordTok{print}\NormalTok{(}\StringTok{'2º Curtil:'}\NormalTok{)}
\end{Highlighting}
\end{Shaded}

\begin{verbatim}
## [1] "2º Curtil:"
\end{verbatim}

\begin{Shaded}
\begin{Highlighting}[]
\KeywordTok{quantile}\NormalTok{(}\KeywordTok{which}\NormalTok{(sentencias_1c}\OperatorTok{$}\NormalTok{reccap }\OperatorTok{!=}\StringTok{ }\DecValTok{0}\NormalTok{),.}\DecValTok{50}\NormalTok{)}
\end{Highlighting}
\end{Shaded}

\begin{verbatim}
##    50% 
## 6271.5
\end{verbatim}

\begin{Shaded}
\begin{Highlighting}[]
\KeywordTok{print}\NormalTok{(}\StringTok{'3º Curtil:'}\NormalTok{)}
\end{Highlighting}
\end{Shaded}

\begin{verbatim}
## [1] "3º Curtil:"
\end{verbatim}

\begin{Shaded}
\begin{Highlighting}[]
\KeywordTok{quantile}\NormalTok{(}\KeywordTok{which}\NormalTok{(sentencias_1c}\OperatorTok{$}\NormalTok{reccap }\OperatorTok{!=}\StringTok{ }\DecValTok{0}\NormalTok{),.}\DecValTok{75}\NormalTok{)}
\end{Highlighting}
\end{Shaded}

\begin{verbatim}
##     75% 
## 8948.75
\end{verbatim}

\begin{Shaded}
\begin{Highlighting}[]
\NormalTok{capmedio <-}\StringTok{ }\KeywordTok{mean}\NormalTok{(}\KeywordTok{pull}\NormalTok{(sentencias_1c[,}\StringTok{'reccap'}\NormalTok{]))}

\NormalTok{sentencias_1c <-}\StringTok{ }\NormalTok{sentencias_1c }\OperatorTok\StringTok{ }
\StringTok{  }\KeywordTok{mutate}\NormalTok{(}\DataTypeTok{reccap_0 =}\NormalTok{ reccap }\OperatorTok{==}\StringTok{ }\DecValTok{0}\NormalTok{) }\OperatorTok\StringTok{ }
\StringTok{  }\KeywordTok{mutate}\NormalTok{(}\DataTypeTok{reccap_1 =}\NormalTok{ (reccap }\OperatorTok{<}\StringTok{ }\KeywordTok{quantile}\NormalTok{(}\KeywordTok{which}\NormalTok{(sentencias_1c}\OperatorTok{$}\NormalTok{reccap }\OperatorTok{!=}\StringTok{ }\DecValTok{0}\NormalTok{),.}\DecValTok{25}\NormalTok{)) }\OperatorTok{&}\StringTok{ }\NormalTok{(reccap}\OperatorTok{!=}\DecValTok{0}\NormalTok{)) }\OperatorTok\StringTok{ }
\StringTok{  }\KeywordTok{mutate}\NormalTok{(}\DataTypeTok{reccap_2 =}\NormalTok{ (reccap }\OperatorTok{>=}\StringTok{ }\KeywordTok{quantile}\NormalTok{(}\KeywordTok{which}\NormalTok{(sentencias_1c}\OperatorTok{$}\NormalTok{reccap }\OperatorTok{!=}\StringTok{ }\DecValTok{0}\NormalTok{),.}\DecValTok{25}\NormalTok{)) }\OperatorTok{&}\StringTok{ }\NormalTok{(reccap }\OperatorTok{<}\StringTok{ }\KeywordTok{quantile}\NormalTok{(}\KeywordTok{which}\NormalTok{(sentencias_1c}\OperatorTok{$}\NormalTok{reccap }\OperatorTok{!=}\StringTok{ }\DecValTok{0}\NormalTok{),.}\DecValTok{50}\NormalTok{))) }\OperatorTok\StringTok{ }
\StringTok{  }\KeywordTok{mutate}\NormalTok{(}\DataTypeTok{reccap_3 =}\NormalTok{ (reccap }\OperatorTok{>=}\StringTok{ }\KeywordTok{quantile}\NormalTok{(}\KeywordTok{which}\NormalTok{(sentencias_1c}\OperatorTok{$}\NormalTok{reccap }\OperatorTok{!=}\StringTok{ }\DecValTok{0}\NormalTok{),.}\DecValTok{50}\NormalTok{)) }\OperatorTok{&}\StringTok{ }\NormalTok{(reccap }\OperatorTok{<}\StringTok{ }\KeywordTok{quantile}\NormalTok{(}\KeywordTok{which}\NormalTok{(sentencias_1c}\OperatorTok{$}\NormalTok{reccap }\OperatorTok{!=}\StringTok{ }\DecValTok{0}\NormalTok{),.}\DecValTok{75}\NormalTok{))) }\OperatorTok\StringTok{ }
\StringTok{  }\KeywordTok{mutate}\NormalTok{(}\DataTypeTok{reccap_4 =}\NormalTok{ (reccap }\OperatorTok{>=}\StringTok{ }\KeywordTok{quantile}\NormalTok{(}\KeywordTok{which}\NormalTok{(sentencias_1c}\OperatorTok{$}\NormalTok{reccap }\OperatorTok{!=}\StringTok{ }\DecValTok{0}\NormalTok{),.}\DecValTok{75}\NormalTok{))) }
\end{Highlighting}
\end{Shaded}

Separo la columna justiciables en 4 rangos para poder aplicar apriori.

\begin{Shaded}
\begin{Highlighting}[]
\NormalTok{sentencias_1c <-}\StringTok{ }\NormalTok{sentencias_1c }\OperatorTok\StringTok{ }
\StringTok{  }\KeywordTok{mutate}\NormalTok{(}\DataTypeTok{justiciables0_1 =}\NormalTok{ justiciables }\OperatorTok{<}\StringTok{ }\DecValTok{2}\NormalTok{) }\OperatorTok\StringTok{ }
\StringTok{  }\KeywordTok{mutate}\NormalTok{(}\DataTypeTok{justiciables2_3 =}\NormalTok{ (justiciables }\OperatorTok{>}\StringTok{ }\DecValTok{1}\NormalTok{) }\OperatorTok{&}\StringTok{ }\NormalTok{(justiciables }\OperatorTok{<}\StringTok{ }\DecValTok{4}\NormalTok{)) }\OperatorTok\StringTok{ }
\StringTok{  }\KeywordTok{mutate}\NormalTok{(}\DataTypeTok{justiciables4_5 =}\NormalTok{ (justiciables }\OperatorTok{>}\StringTok{ }\DecValTok{3}\NormalTok{) }\OperatorTok{&}\StringTok{ }\NormalTok{(justiciables }\OperatorTok{<}\StringTok{ }\DecValTok{6}\NormalTok{)) }\OperatorTok\StringTok{ }
\StringTok{  }\KeywordTok{mutate}\NormalTok{(}\DataTypeTok{justiciables6_7 =}\NormalTok{ (justiciables }\OperatorTok{>}\StringTok{ }\DecValTok{5}\NormalTok{) }\OperatorTok{&}\StringTok{ }\NormalTok{(justiciables }\OperatorTok{<}\StringTok{ }\DecValTok{8}\NormalTok{)) }\OperatorTok\StringTok{ }
\StringTok{  }\KeywordTok{mutate}\NormalTok{(}\DataTypeTok{justiciables8_9 =}\NormalTok{ (justiciables }\OperatorTok{>}\StringTok{ }\DecValTok{7}\NormalTok{) }\OperatorTok{&}\StringTok{ }\NormalTok{(justiciables }\OperatorTok{<}\StringTok{ }\DecValTok{10}\NormalTok{)) }\OperatorTok\StringTok{ }
\StringTok{  }\KeywordTok{mutate}\NormalTok{(}\DataTypeTok{justiciables10_N =}\NormalTok{ justiciables }\OperatorTok{>}\StringTok{ }\DecValTok{9}\NormalTok{)}
\end{Highlighting}
\end{Shaded}

Separamos Localidad y Circunscripcion en columnas.

\begin{Shaded}
\begin{Highlighting}[]
\NormalTok{sentencias_1c <-}\StringTok{ }\NormalTok{sentencias_1c }\OperatorTok\StringTok{ }
\StringTok{  }\KeywordTok{mutate}\NormalTok{(}\DataTypeTok{localidad =} \KeywordTok{as.factor}\NormalTok{(localidad))}
\end{Highlighting}
\end{Shaded}

Convertimos columna tproc en categórica, esto es requerido por el
algoritmo

\begin{Shaded}
\begin{Highlighting}[]
\NormalTok{sentencias_1c <-}\StringTok{ }\NormalTok{sentencias_1c }\OperatorTok\StringTok{ }
\StringTok{  }\KeywordTok{mutate}\NormalTok{(}\DataTypeTok{tproc =} \KeywordTok{as.factor}\NormalTok{(tproc)) }\OperatorTok\StringTok{ }
\StringTok{  }\KeywordTok{mutate}\NormalTok{(}\DataTypeTok{circunscripcion =} \KeywordTok{as.factor}\NormalTok{(circunscripcion)) }\OperatorTok\StringTok{ }
\StringTok{  }\KeywordTok{mutate}\NormalTok{(}\DataTypeTok{materia =} \KeywordTok{as.factor}\NormalTok{(materia))}
\end{Highlighting}
\end{Shaded}

Tomos solamente las columnas tipo booleanos y categóricas.

\begin{Shaded}
\begin{Highlighting}[]
\NormalTok{sentencias_final <-}\StringTok{ }\NormalTok{sentencias_1c }\OperatorTok\StringTok{ }
\StringTok{  }\KeywordTok{select}\NormalTok{(}\OperatorTok{-}\NormalTok{nro, }\OperatorTok{-}\NormalTok{as, }\OperatorTok{-}\NormalTok{ccon, }\OperatorTok{-}\NormalTok{finicio, }\OperatorTok{-}\NormalTok{fres, }\OperatorTok{-}\NormalTok{fdesp, }\OperatorTok{-}\NormalTok{fvenc1, }\OperatorTok{-}\NormalTok{fvenc2, }\OperatorTok{-}\NormalTok{tres, }\OperatorTok{-}\NormalTok{justiciables, }\OperatorTok{-}\NormalTok{reccap, }\OperatorTok{-}\NormalTok{iep, }\OperatorTok{-}\NormalTok{localidad)}

\NormalTok{sentencias_final }\OperatorTok\StringTok{ }
\StringTok{  }\KeywordTok{mostrar}\NormalTok{(}\DataTypeTok{caption =} \StringTok{"Tabla final a utilizar en el algoritmo apriori"}\NormalTok{)}
\end{Highlighting}
\end{Shaded}

\rowcolors{2}{gray!6}{white}

\begin{table}

\caption{\label{tab:unnamed-chunk-17}Tabla final a utilizar en el algoritmo apriori}
\centering
\resizebox{\linewidth}{!}{
\begin{tabular}[t]{l|l|l|l|l|l|l|l|l|l|l|l|l|l|l|l|l}
\hiderowcolors
\hline
tproc & rapido & normal & demorado & circunscripcion & materia & reccap\_0 & reccap\_1 & reccap\_2 & reccap\_3 & reccap\_4 & justiciables0\_1 & justiciables2\_3 & justiciables4\_5 & justiciables6\_7 & justiciables8\_9 & justiciables10\_N\\
\hline
\showrowcolors
RESTRICCIONES A LA CAPACIDAD & FALSE & TRUE & FALSE & Gualeguaychú & fam|pen & TRUE & FALSE & FALSE & FALSE & FALSE & TRUE & FALSE & FALSE & FALSE & FALSE & FALSE\\
\hline
INCIDENTE & FALSE & FALSE & TRUE & Gualeguaychú & fam|pen & TRUE & FALSE & FALSE & FALSE & FALSE & FALSE & TRUE & FALSE & FALSE & FALSE & FALSE\\
\hline
ORDINARIO FILIACION E INDEMNIZACION DE DAÑOS & FALSE & TRUE & FALSE & Gualeguaychú & fam|pen & TRUE & FALSE & FALSE & FALSE & FALSE & TRUE & FALSE & FALSE & FALSE & FALSE & FALSE\\
\hline
MEDIDA CAUTELAR (FAMILIA) & FALSE & FALSE & TRUE & Gualeguaychú & fam|pen & TRUE & FALSE & FALSE & FALSE & FALSE & TRUE & FALSE & FALSE & FALSE & FALSE & FALSE\\
\hline
ORDINARIO DAÑOS Y PERJUICIOS & FALSE & TRUE & FALSE & Gualeguaychú & fam|pen & TRUE & FALSE & FALSE & FALSE & FALSE & FALSE & TRUE & FALSE & FALSE & FALSE & FALSE\\
\hline
ORDINARIO FILIACION E INDEMNIZACION DE DAÑOS & FALSE & FALSE & TRUE & Gualeguaychú & fam|pen & TRUE & FALSE & FALSE & FALSE & FALSE & FALSE & TRUE & FALSE & FALSE & FALSE & FALSE\\
\hline
\end{tabular}}
\end{table}

\rowcolors{2}{white}{white}

\begin{Shaded}
\begin{Highlighting}[]
\NormalTok{rules <-}\StringTok{ }\KeywordTok{apriori}\NormalTok{(sentencias_final, }\DataTypeTok{parameter =} \KeywordTok{list}\NormalTok{(}\DataTypeTok{supp=}\FloatTok{0.01}\NormalTok{, }\DataTypeTok{conf=}\FloatTok{0.7}\NormalTok{, }\DataTypeTok{minlen=}\DecValTok{2}\NormalTok{), }\DataTypeTok{appearance =} \KeywordTok{list}\NormalTok{(}\DataTypeTok{rhs=}\KeywordTok{c}\NormalTok{(}\StringTok{"demorado"}\NormalTok{, }\StringTok{"rapido"}\NormalTok{)))}
\end{Highlighting}
\end{Shaded}

\begin{verbatim}
## Apriori
## 
## Parameter specification:
##  confidence minval smax arem  aval originalSupport maxtime support minlen
##         0.7    0.1    1 none FALSE            TRUE       5    0.01      2
##  maxlen target   ext
##      10  rules FALSE
## 
## Algorithmic control:
##  filter tree heap memopt load sort verbose
##     0.1 TRUE TRUE  FALSE TRUE    2    TRUE
## 
## Absolute minimum support count: 115 
## 
## set item appearances ...[2 item(s)] done [0.00s].
## set transactions ...[275 item(s), 11576 transaction(s)] done [0.00s].
## sorting and recoding items ... [46 item(s)] done [0.00s].
## creating transaction tree ... done [0.00s].
## checking subsets of size 1 2 3 4 5 6 done [0.00s].
## writing ... [8 rule(s)] done [0.00s].
## creating S4 object  ... done [0.00s].
\end{verbatim}

\begin{Shaded}
\begin{Highlighting}[]
\KeywordTok{summary}\NormalTok{(rules)}
\end{Highlighting}
\end{Shaded}

\begin{verbatim}
## set of 8 rules
## 
## rule length distribution (lhs + rhs):sizes
## 3 4 5 6 
## 1 3 3 1 
## 
##    Min. 1st Qu.  Median    Mean 3rd Qu.    Max. 
##     3.0     4.0     4.5     4.5     5.0     6.0 
## 
## summary of quality measures:
##     support          confidence          lift           count      
##  Min.   :0.01002   Min.   :0.7134   Min.   :2.674   Min.   :116.0  
##  1st Qu.:0.01002   1st Qu.:0.7147   1st Qu.:2.679   1st Qu.:116.0  
##  Median :0.01006   Median :0.7367   Median :2.762   Median :116.5  
##  Mean   :0.01009   Mean   :0.7362   Mean   :2.760   Mean   :116.8  
##  3rd Qu.:0.01013   3rd Qu.:0.7582   3rd Qu.:2.842   3rd Qu.:117.2  
##  Max.   :0.01019   Max.   :0.7582   Max.   :2.842   Max.   :118.0  
## 
## mining info:
##              data ntransactions support confidence
##  sentencias_final         11576    0.01        0.7
\end{verbatim}

\begin{Shaded}
\begin{Highlighting}[]
\KeywordTok{inspect}\NormalTok{(rules[}\DecValTok{1}\OperatorTok{:}\DecValTok{8}\NormalTok{])}
\end{Highlighting}
\end{Shaded}

\begin{verbatim}
##     lhs                          rhs           support confidence     lift count
## [1] {circunscripcion=Uruguay,                                                   
##      reccap_1}                => {demorado} 0.01019350  0.7151515 2.680892   118
## [2] {tproc=APREMIO,                                                             
##      circunscripcion=Uruguay,                                                   
##      reccap_1}                => {demorado} 0.01002073  0.7581699 2.842155   116
## [3] {circunscripcion=Uruguay,                                                   
##      materia=paz,                                                               
##      reccap_1}                => {demorado} 0.01019350  0.7151515 2.680892   118
## [4] {circunscripcion=Uruguay,                                                   
##      reccap_1,                                                                  
##      justiciables2_3}         => {demorado} 0.01010712  0.7134146 2.674381   117
## [5] {tproc=APREMIO,                                                             
##      circunscripcion=Uruguay,                                                   
##      materia=paz,                                                               
##      reccap_1}                => {demorado} 0.01002073  0.7581699 2.842155   116
## [6] {tproc=APREMIO,                                                             
##      circunscripcion=Uruguay,                                                   
##      reccap_1,                                                                  
##      justiciables2_3}         => {demorado} 0.01002073  0.7581699 2.842155   116
## [7] {circunscripcion=Uruguay,                                                   
##      materia=paz,                                                               
##      reccap_1,                                                                  
##      justiciables2_3}         => {demorado} 0.01010712  0.7134146 2.674381   117
## [8] {tproc=APREMIO,                                                             
##      circunscripcion=Uruguay,                                                   
##      materia=paz,                                                               
##      reccap_1,                                                                  
##      justiciables2_3}         => {demorado} 0.01002073  0.7581699 2.842155   116
\end{verbatim}

\begin{Shaded}
\begin{Highlighting}[]
\NormalTok{top.confidence <-}\StringTok{ }\KeywordTok{sort}\NormalTok{(rules, }\DataTypeTok{decreasing =} \OtherTok{TRUE}\NormalTok{, }\DataTypeTok{na.last =} \OtherTok{NA}\NormalTok{, }\DataTypeTok{by =} \StringTok{"confidence"}\NormalTok{)}
\KeywordTok{inspect}\NormalTok{(top.confidence[}\DecValTok{1}\OperatorTok{:}\DecValTok{8}\NormalTok{])}
\end{Highlighting}
\end{Shaded}

\begin{verbatim}
##     lhs                          rhs           support confidence     lift count
## [1] {tproc=APREMIO,                                                             
##      circunscripcion=Uruguay,                                                   
##      reccap_1}                => {demorado} 0.01002073  0.7581699 2.842155   116
## [2] {tproc=APREMIO,                                                             
##      circunscripcion=Uruguay,                                                   
##      materia=paz,                                                               
##      reccap_1}                => {demorado} 0.01002073  0.7581699 2.842155   116
## [3] {tproc=APREMIO,                                                             
##      circunscripcion=Uruguay,                                                   
##      reccap_1,                                                                  
##      justiciables2_3}         => {demorado} 0.01002073  0.7581699 2.842155   116
## [4] {tproc=APREMIO,                                                             
##      circunscripcion=Uruguay,                                                   
##      materia=paz,                                                               
##      reccap_1,                                                                  
##      justiciables2_3}         => {demorado} 0.01002073  0.7581699 2.842155   116
## [5] {circunscripcion=Uruguay,                                                   
##      reccap_1}                => {demorado} 0.01019350  0.7151515 2.680892   118
## [6] {circunscripcion=Uruguay,                                                   
##      materia=paz,                                                               
##      reccap_1}                => {demorado} 0.01019350  0.7151515 2.680892   118
## [7] {circunscripcion=Uruguay,                                                   
##      reccap_1,                                                                  
##      justiciables2_3}         => {demorado} 0.01010712  0.7134146 2.674381   117
## [8] {circunscripcion=Uruguay,                                                   
##      materia=paz,                                                               
##      reccap_1,                                                                  
##      justiciables2_3}         => {demorado} 0.01010712  0.7134146 2.674381   117
\end{verbatim}

\begin{Shaded}
\begin{Highlighting}[]
\NormalTok{top.support <-}\StringTok{ }\KeywordTok{sort}\NormalTok{(rules, }\DataTypeTok{decreasing =} \OtherTok{TRUE}\NormalTok{, }\DataTypeTok{na.last =} \OtherTok{NA}\NormalTok{, }\DataTypeTok{by =} \StringTok{"support"}\NormalTok{)}
\KeywordTok{inspect}\NormalTok{(top.support[}\DecValTok{1}\OperatorTok{:}\DecValTok{8}\NormalTok{])}
\end{Highlighting}
\end{Shaded}

\begin{verbatim}
##     lhs                          rhs           support confidence     lift count
## [1] {circunscripcion=Uruguay,                                                   
##      reccap_1}                => {demorado} 0.01019350  0.7151515 2.680892   118
## [2] {circunscripcion=Uruguay,                                                   
##      materia=paz,                                                               
##      reccap_1}                => {demorado} 0.01019350  0.7151515 2.680892   118
## [3] {circunscripcion=Uruguay,                                                   
##      reccap_1,                                                                  
##      justiciables2_3}         => {demorado} 0.01010712  0.7134146 2.674381   117
## [4] {circunscripcion=Uruguay,                                                   
##      materia=paz,                                                               
##      reccap_1,                                                                  
##      justiciables2_3}         => {demorado} 0.01010712  0.7134146 2.674381   117
## [5] {tproc=APREMIO,                                                             
##      circunscripcion=Uruguay,                                                   
##      reccap_1}                => {demorado} 0.01002073  0.7581699 2.842155   116
## [6] {tproc=APREMIO,                                                             
##      circunscripcion=Uruguay,                                                   
##      materia=paz,                                                               
##      reccap_1}                => {demorado} 0.01002073  0.7581699 2.842155   116
## [7] {tproc=APREMIO,                                                             
##      circunscripcion=Uruguay,                                                   
##      reccap_1,                                                                  
##      justiciables2_3}         => {demorado} 0.01002073  0.7581699 2.842155   116
## [8] {tproc=APREMIO,                                                             
##      circunscripcion=Uruguay,                                                   
##      materia=paz,                                                               
##      reccap_1,                                                                  
##      justiciables2_3}         => {demorado} 0.01002073  0.7581699 2.842155   116
\end{verbatim}

\begin{Shaded}
\begin{Highlighting}[]
\KeywordTok{library}\NormalTok{(arulesViz)}
\end{Highlighting}
\end{Shaded}

\begin{verbatim}
## Loading required package: grid
\end{verbatim}

\begin{Shaded}
\begin{Highlighting}[]
\KeywordTok{plot}\NormalTok{(rules, }\DataTypeTok{method =} \StringTok{"grouped"}\NormalTok{)}
\end{Highlighting}
\end{Shaded}

\includegraphics[height=0.4\textheight]{TPMineria_files/figure-latex/unnamed-chunk-19-1}

\end{document}
